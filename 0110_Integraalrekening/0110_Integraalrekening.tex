\documentclass[a4paper,12pt]{article}

\textwidth 17cm \textheight 25cm \evensidemargin 0cm
\oddsidemargin 0cm \topmargin -2cm
\parindent 0pt
%\parskip \bigskipamount

\usepackage{graphicx}
\usepackage[dutch]{babel}
\usepackage{amssymb,amsthm,amsmath}
%\usepackage{dot2texi}
\usepackage[utf8]{inputenc}
\usepackage{nopageno}
\usepackage{pdfpages}
\usepackage{enumerate}
\usepackage{caption}
\usepackage{wrapfig}
\usepackage{pgf,tikz,pgfplots}
\pgfplotsset{compat=1.15}
\usepackage{color}
\usetikzlibrary{arrows}
\usetikzlibrary{patterns}
\usepackage{fancyhdr}
\pagestyle{fancy}
\usepackage[version=3]{mhchem}
\usepackage{multicol}
\usepackage{fix-cm}
\usepackage{setspace}
\usepackage{mhchem}
\usepackage{xhfill}
\usepackage{parskip}
\usepackage{cancel}
\usepackage{mdframed}
\usepackage{url}
\usepackage{mathtools}
\usepackage{changepage}

\newcommand{\todo}[1]{{\color{red} TODO: #1}}

\newcommand{\degree}{\ensuremath{^\circ}}
\newcommand\rad{\qopname\relax o{\mathrm{rad}}}

\newcommand\ggd{\qopname\relax o{\mathrm{ggd}}}

\pgfmathdeclarefunction{gauss}{2}{%
  \pgfmathparse{1/(#2*sqrt(2*pi))*exp(-((x-#1)^2)/(2*#2^2))}%
}

\def\LRA{\Leftrightarrow}

\newcommand{\zrmbox}{\framebox{\phantom{EXE}}\phantom{X}}
\newcommand{\zrm}[1]{\framebox{#1}}

% environment oefening:
% houdt een teller bij die de oefeningen nummert, probeert ook de oefening op één pagina te houden
\newcounter{noefening}
\setcounter{noefening}{0}
\newenvironment{oefening}
{
  \stepcounter{noefening}
  \pagebreak[0]
  \begin{minipage}{\textwidth}
  \vspace*{0.7cm}{\large\bf Oefening \arabic{noefening}}
}{%
  \end{minipage}
}

\usepackage{calc}

% vraag
\reversemarginpar
\newcounter{punten}
\setcounter{punten}{0}
\newcounter{nvraag}
\setcounter{nvraag}{1}
\newlength{\puntwidth}
\newlength{\boxwidth}
\newcommand{\vraag}[1]{
\settowidth{\puntwidth}{\Large{#1}}
\setlength{\boxwidth}{1.5cm}
\addtolength{\boxwidth}{-\puntwidth}
{\large\bf Vraag \arabic{nvraag} \addtocounter{nvraag}{1}}\vspace*{-0.5cm}
{\marginpar{\color{lightgray}\fbox{\parbox{1.5cm}{\vspace*{1cm}\hspace*{\boxwidth}{\Large{#1}}}}}
\vspace*{0.5cm}}
\addtocounter{punten}{#1}}

% arulefill
\def\arulefill{\leavevmode{\xrfill[-5pt]{0.3pt}[lightgray]\endgraf}\vspace*{0.2cm}}

% \arules{n}
\newcommand{\arules}[1]{
\color{lightgray}
%\vspace*{0.05cm}
\foreach \n in {1,...,#1}{
  \vspace*{0.75cm}
  \hrule height 0.3pt\hfill
}\color{black}\vspace*{0.2cm}}

% \arule{x}
\newcommand{\arule}[1]{
\color{lightgray}{\raisebox{-0.1cm}{\rule[-0.05cm]{#1}{0.3pt}}}\color{black}
}

% \abox{y}
\newcommand{\abox}[1]{
\fbox{
\begin{minipage}{\textwidth- 4\fboxsep}
\hspace*{\textwidth}\vspace{#1}
\end{minipage}
}
}

\newcommand{\ruitjes}[1]{
\definecolor{cqcqcq}{rgb}{0.85,0.85,0.85}
\hspace*{-2.5cm}
\begin{tikzpicture}[scale=1.04,line cap=round,line join=round,>=triangle 45,x=1.0cm,y=1.0cm]
\draw [color=cqcqcq, xstep=0.5cm, ystep=0.5cm] (0,-#1) grid (20.5,0);
\end{tikzpicture}
}


\newcommand{\assenstelsel}[5][1]{
\definecolor{cqcqcq}{rgb}{0.65,0.65,0.65}
\begin{tikzpicture}[line cap=round,line join=round,>=triangle 45,x=#1cm,y=#1cm]
\draw [color=cqcqcq,dash pattern=on 1pt off 1pt, xstep=1.0cm,ystep=1.0cm] (#2,#4) grid (#3,#5);
\draw[->,color=black] (#2,0) -- (#3,0);
%\draw[shift={(1,0)},color=black] (0pt,2pt) -- (0pt,-2pt) node[below] {\footnotesize $1$};
%\draw[color=black] (#3.25,0.07) node [anchor=south west] {$x$};
\draw[->,color=black] (0,#4) -- (0,#5);
%\draw[shift={(0,1)},color=black] (2pt,0pt) -- (-2pt,0pt) node[left] {\footnotesize $1$};
\draw[color=black] (0.09,#5.25) node [anchor=west] {\phantom{$y$}};
%\draw[color=black] (0pt,-10pt) node[right] {\footnotesize $0$};
\end{tikzpicture}
}

\newcommand{\getallenas}[3][1]{
\definecolor{cqcqcq}{rgb}{0.65,0.65,0.65}
\begin{tikzpicture}[scale=#1,line cap=round,line join=round,>=triangle 45,x=1.0cm,y=1.0cm]
\draw [color=cqcqcq,dash pattern=on 1pt off 1pt, xstep=1.0cm,ystep=1.0cm] (#2,-0.2) grid (#3,0.2);
\draw[->,color=black] (#2.25,0) -- (#3.5,0);
\draw[shift={(0,0)},color=black] (0pt,2pt) -- (0pt,-2pt) node[below] {\footnotesize $0$};
\draw[shift={(1,0)},color=black] (0pt,2pt) -- (0pt,-2pt) node[below] {\footnotesize $1$};
\draw[color=black] (#3.25,0.07) node [anchor=south west] {$\mathbb{R}$};
\end{tikzpicture}
}

\newcommand{\visgraad}[1]{\begin{tabular}{p{0.5cm}|p{#1}}&\\\hline\\\end{tabular}}

\newcommand{\tekenschema}[2]{\begin{tabular}{p{0.5cm}|p{#1}}&\\\hline\\[#2]\end{tabular}}

% schema van Horner
\newcommand{\schemahorner}{
\begin{tabular}{p{0.5cm}|p{7cm}}
&\\[1.5cm]
\hline\\
\end{tabular}}

% geef tabular iets meer ruimte
\setlength{\tabcolsep}{14pt}
\renewcommand{\arraystretch}{1.5}

\newcommand{\toets}[3]{
\thispagestyle{plain}
\vspace*{-2.5cm}
\begin{tikzpicture}[remember picture, overlay]
    \node [shift={(15.25 cm,-1.6cm)}] {%
        \includegraphics[width=1.8cm]{/home/ppareit/kaa1415/logokaavelgem.png}%
    };%
\end{tikzpicture}

\begin{tabular}{|llc|c|}
\hline
\vspace*{-0.5cm}
&&&\\
Naam & \arule{4cm} & {\Large\bf KA AVELGEM} & \\
\vspace*{-0.75cm}
&&&\\
Klas & \arule{4cm} & {\Large\bf 20...-...-...} & \\
\hline
\vspace*{-0.75cm}
&&&\\
Toets & {\bf #2} & {\large\bf #1} & Beoordeling\\
\vspace*{-0.75cm}
&&&\\
Onderwerp & \multicolumn{2}{l|}{\bf #3} &\\
\hline
\end{tabular}
}

\newcommand{\oefeningen}[1]{

\fancyhead[LE, RO]{\vspace{0.5cm} #1}
%\thispagestyle{plain}

{\bf \Large \centering Oefeningen: #1}

}

\raggedbottom

\newcommand\vl{\qopname\relax o{\mathrm{vl}}}

\newcommand\dom{\qopname\relax o{\mathrm{dom}}}
\newcommand\ber{\qopname\relax o{\mathrm{ber}}}

\newcommand\mC{\qopname\relax o{\mathrm{mC}}}
\newcommand\uC{\qopname\relax o{\mathrm{{\mu}C}}}
\newcommand\C{\qopname\relax o{\mathrm{C}}}

\newcommand\W{\qopname\relax o{\mathrm{W}}}
\newcommand\kW{\qopname\relax o{\mathrm{kW}}}
\newcommand\kWh{\qopname\relax o{\mathrm{kWh}}}


\newcommand\V{\qopname\relax o{\mathrm{V}}}
\newcommand\ohm{\qopname\relax o{\mathrm{\Omega}}}
\newcommand\kohm{\qopname\relax o{\mathrm{k\Omega}}}


\newcommand\N{\qopname\relax o{\mathrm{N}}}

\newcommand\Nperkg{\qopname\relax o{\mathrm{N/kg}}}

\newcommand\Nperm{\qopname\relax o{\mathrm{N/m}}}

\newcommand\gpermol{\qopname\relax o{\mathrm{g/mol}}}


\newcommand\kgperm{\qopname\relax o{\mathrm{kg/m}}}
\newcommand\kgperdm{\qopname\relax o{\mathrm{kg/dm}}}
\newcommand\gpercm{\qopname\relax o{\mathrm{g/cm}}}
\newcommand\gperml{\qopname\relax o{\mathrm{g/ml}}}


\newcommand{\mA}{\;\mbox{mA}}
\newcommand{\A}{\;\mbox{A}}
\newcommand{\MA}{\;\mbox{MA}}

\newcommand{\us}{\;\mu\mbox{s}}
\newcommand\s{\qopname\relax o{\mathrm{s}}}

\newcommand\h{\qopname\relax o{\mathrm{h}}}

\newcommand{\kmperh}{\;\mbox{km/h}}
\newcommand{\mpers}{\;\mbox{m/s}}
\newcommand{\kmpermin}{\;\mbox{km/min}}
\newcommand{\kmpers}{\;\mbox{km/s}}

\newcommand{\mph}{\;\mbox{mph}}

\newcommand{\Hz}{\;\mbox{Hz}}

\newcommand\Gm{\qopname\relax o{\mathrm{Gm}}}
\newcommand\Mm{\qopname\relax o{\mathrm{Mm}}}
\newcommand\km{\qopname\relax o{\mathrm{km}}}
\newcommand\hm{\qopname\relax o{\mathrm{hm}}}
\newcommand\dam{\qopname\relax o{\mathrm{dam}}}
\newcommand\m{\qopname\relax o{\mathrm{m}}}
\newcommand\dm{\qopname\relax o{\mathrm{dm}}}
\newcommand\cm{\qopname\relax o{\mathrm{cm}}}
\newcommand\mm{\qopname\relax o{\mathrm{mm}}}
\newcommand\um{\qopname\relax o{\mathrm{{\mu}m}}}
\newcommand\nm{\qopname\relax o{\mathrm{nm}}}


\newcommand\Gg{\qopname\relax o{\mathrm{Gg}}}
\newcommand\Mg{\qopname\relax o{\mathrm{Mg}}}
\newcommand\kg{\qopname\relax o{\mathrm{kg}}}
\newcommand\hg{\qopname\relax o{\mathrm{hg}}}
\renewcommand\dag{\qopname\relax o{\mathrm{dag}}}
\newcommand\g{\qopname\relax o{\mathrm{g}}}
\newcommand\dg{\qopname\relax o{\mathrm{dg}}}
\newcommand\cg{\qopname\relax o{\mathrm{cg}}}
\newcommand\mg{\qopname\relax o{\mathrm{mg}}}
\newcommand\ug{\qopname\relax o{\mathrm{{\mu}g}}}
\renewcommand\ng{\qopname\relax o{\mathrm{ng}}}

\newcommand\ton{\qopname\relax o{\mathrm{ton}}}

\newcommand\Gl{\qopname\relax o{\mathrm{Gl}}}
\newcommand\Ml{\qopname\relax o{\mathrm{Ml}}}
\newcommand\kl{\qopname\relax o{\mathrm{kl}}}
\newcommand\hl{\qopname\relax o{\mathrm{hl}}}
\newcommand\dal{\qopname\relax o{\mathrm{dal}}}
\renewcommand\l{\qopname\relax o{\mathrm{l}}}
\newcommand\dl{\qopname\relax o{\mathrm{dl}}}
\newcommand\cl{\qopname\relax o{\mathrm{cl}}}
\newcommand\ml{\qopname\relax o{\mathrm{ml}}}
\newcommand\ul{\qopname\relax o{\mathrm{{\mu}l}}}
\newcommand\nl{\qopname\relax o{\mathrm{nl}}}

\newcommand\MJ{\qopname\relax o{\mathrm{MJ}}}
\newcommand\kJ{\qopname\relax o{\mathrm{kJ}}}
\newcommand\J{\qopname\relax o{\mathrm{J}}}

\newcommand\T{\qopname\relax o{\mathrm{T}}}
\newcommand\uT{\qopname\relax o{\mathrm{{\mu}T}}}

\newcommand\grC{\qopname\relax o{\mathrm{{\degree}C}}}

\newcommand\K{\qopname\relax o{\mathrm{K}}}
\newcommand\calperK{\qopname\relax o{\mathrm{cal/K}}}

\newcommand\hPa{\qopname\relax o{\mathrm{hPa}}}
\newcommand\Pa{\qopname\relax o{\mathrm{Pa}}}

\newcommand\dB{\qopname\relax o{\mathrm{dB}}}

\newcommand\Var{\qopname\relax o{\mathrm{Var}}}

\newcommand{\EE}[1]{\cdot 10^{#1}}

\onehalfspacing

%\setlength{\headsep}{0cm}

\newenvironment{exlist}[1] %
{ \begin{multicols}{#1}
  \begin{enumerate}[(a)]
    \setlength{\itemsep}{0.5em} }
{ \end{enumerate}
  \end{multicols} }




\usepackage{versions}
%\includeversion{theorie}
\excludeversion{theorie}

\begin{document}

\pagestyle{fancy}
\lhead{}
\rhead{Oefeningen Integraalrekening}

\begin{theorie}

\thispagestyle{empty}
\begin{center}
  \begin{mdframed}
  \centering
  \fontsize{40}{50}\selectfont Integraalrekening
  \end{mdframed}
  \vfill
  \vfill
\end{center}
\subsection*{Doelstelling}
Je \hfill  {\scriptsize(LP 2006-059, LI 1.10)}
\begin{itemize}
\end{itemize}


\pagestyle{empty}
\mbox{}
\newpage
\clearpage
\thispagestyle{empty}
%\mbox{}
{\small \tableofcontents}
\newpage
\clearpage
\pagenumbering{arabic}

\pagestyle{fancy}
\lhead{}
\rhead{Goniometrische formules}

\end{theorie}

\onehalfspacing

\section{Differentiaal van een functie}


\begin{oefening}
Bepaal de differentiaal van de volgende functies
\begin{enumerate}[(a)]
\itemsep.5em
  \item $\displaystyle 5+3x^2+8x^4$
  \item $\displaystyle (x^2+x^4+x^6)^4$
  \item $\displaystyle \sqrt{x^2+2x}$
  \item $\displaystyle \dfrac{x^3+3x^2+3x+1}{x+1}$
  \item $\displaystyle e^{\sqrt[3]{3x}}$
  \item $\displaystyle \dfrac{1}{\ln 10}\cdot 10^{x^2}$
  \item $x\sin x + \cos x$
\end{enumerate}
\end{oefening}

\begin{oefening}
Benader
\begin{enumerate}[(a)]
\itemsep.5em
  \item $\displaystyle \sqrt[3]{9}$
  \item $\displaystyle \sqrt{24}$
  \item $\displaystyle e^{0.2}$
  \item $\displaystyle \sqrt{9.5}$
\end{enumerate}
\end{oefening}

\begin{oefening} % bron http://info.math4all.nl/MathAdore/vb-bb43-ex2b.html
De luchtdruk $p$ (in hectopascal $\hPa$) hangt af van de hoogte $h$ in $\km$ boven het aardoppervlak. In een luchtballon is de luchtdruk gemakkelijk te meten en wordt daaruit de hoogte berekend met de formule:
$$h = -6.5 \log \frac{p}{p_0}$$
Hierin is $p_0$ de luchtdruk op zeeniveau. Neem aan dat $p_0 = 1000 \hPa$.
Bereken nu de hoogte en de snelheid waarmee $h(p)$ verandert als $p = 920 \hPa$ wordt gemeten. 
\end{oefening}

\pagebreak
\section{Bepaalde integraal}

\begin{oefening}
Bereken volgende bepaalde integralen
\begin{multicols}{2}
\begin{enumerate}[(a)]
\itemsep1em
  \item $\displaystyle \int_0^4 x \;dx$
  \item $\displaystyle \int_1^3 x^4 \;dx$
  \item $\displaystyle \int_{-4}^4 \;dx$
  \item $\displaystyle \int_{-1}^2 (3x^2+2x-1)\;dx$
  \item $\displaystyle \int_{-4}^3 (6x^8+12x^5-\dfrac{1}{2}x^2)\;dx$
  \item $\displaystyle \int_{0}^5 (3x+5)^2\;dx$
  \item $\displaystyle \int_1^2 \dfrac{1}{x^2}\;dx$
  \item $\displaystyle \int_{-3}^{-1} \dfrac{4}{3x^3} \;dx$
  \item $\displaystyle \int_4^{16} \sqrt{x} \;dx$
  \item $\displaystyle \int_8^{27} \sqrt[3]{x} \;dx$
  \item $\displaystyle \int_2^8 \sqrt{2x} \;dx$
  \item $\displaystyle \int_8^{27} \dfrac{1}{\sqrt[3]{x}} \;dx$
  \item $\displaystyle \int_3^{16/3} \dfrac{x}{\sqrt{3x}} \;dx$
  \item $\displaystyle \int_{0}^3 (x+1)^3\;dx$
  \item $\displaystyle \int_{0}^2 (x+2)^6\;dx$
  \item $\displaystyle \int_{-1}^3 \dfrac{x+1}{x}\;dx$
  \item $\displaystyle \int_{-4}^4 \dfrac{x^2+2x+2}{4x^2}\;dx$
  \item $\displaystyle \int_{4}^0 \sqrt{x}(x-2)\;dx$
\end{enumerate}
\end{multicols}
\end{oefening}

\begin{oefening}
Bereken
\begin{multicols}{2}
\begin{enumerate}[(a)]
\itemsep1em
  \item $\displaystyle \int_1^2 y^2+y^{-2} \;dy$
  \item $\displaystyle \int_{-1}^2 y^2+y^{-2} \;dy$
  \item $\displaystyle \int_1^2 \dfrac{2\omega^5 - \omega + 3}{\omega^2} \;d\omega$
  \item $\displaystyle \int_{25}^{-10} \;dR$
\end{enumerate}
\end{multicols}
\end{oefening}

\begin{oefening} % source http://tutorial.math.lamar.edu/Classes/CalcI/ComputingDefiniteIntegrals.aspx
Gegeven
$$f(x)=\begin{cases}6 & \mbox{ if } x > 1\\3x^2 & \mbox{ if } x \leq 1\\\end{cases}$$
Bepaal
$$\displaystyle \int_{10}^{22} f(x) \;dx \qquad\mbox{ en }\qquad \displaystyle \int_{-2}^{3} f(x) \;dx$$
\end{oefening}

\section{Oppervlakteberekeningen}

\begin{oefening}
Bepaal de oppervlakte tussen
\begin{enumerate}[(a)]
\itemsep1em
  \item $\displaystyle y=x^2,\quad y=0,\quad x=0, \quad x=2$
  \item $\displaystyle y=x^2-6x,\quad y=0,\quad x=2, \quad x=4$
  \item $\displaystyle y=\sqrt{x},\quad y=0,\quad x=0, \quad x=9$
  \item $\displaystyle y=\sqrt[3]{x},\quad y=0,\quad x=0, \quad x=8$
  \item $\displaystyle y=\dfrac{x-1}{x},\quad y=0,\quad x=-3, \quad x=-1$
  \item $\displaystyle y=\dfrac{x^2-4}{x^2},\quad y=0,\quad x=1, \quad x=6$
\end{enumerate}
\end{oefening}

\begin{oefening}
Bepaal de oppervlakte tussen
\begin{enumerate}[(a)]
\itemsep1em
  \item $\displaystyle y=x(x-3),\quad y=0,\quad x=0, \quad x=5$
  \item $\displaystyle y=\sqrt{x},\quad y=\sqrt{x+1},\quad x=0, \quad x=4$
\end{enumerate}
\end{oefening}


\begin{oefening}
Bepaal de oppervlakte tussen de gegeven kromme en de $x$-as
\begin{multicols}{2}
\begin{enumerate}[(a)]
\itemsep1em
  \item $\displaystyle y=4x-x^2$
  \item $\displaystyle y=4x-x^3$
  \item $\displaystyle y=-x^3+5x^2-7x+3$
  \item $\displaystyle y=4x^4-8x^3-12x^2+16x+16$
\end{enumerate}
\end{multicols}
\end{oefening}

\begin{oefening}
Bepaal de oppervlakte tussen $f$ en $g$
\begin{enumerate}[(a)]
\itemsep1em
  \item $f(x)=x^2$ en $g(x)=2x$
  \item $f(x)=x$ en $g(x)=2x-x^2$
\end{enumerate}
\end{oefening}

\begin{oefening}
Bepaal de oppervlakte tussen
\begin{enumerate}[(a)]
\itemsep1em
  \item $x=y^3-26y+10$ en $x=40-6y^2-y^3$
  \item $x=|y|$ en $x=1-|y|$
  \item $y=|x|$ en $y=x^2-6$
  \item $y=\sqrt{x}$ en $y=x^2$
\end{enumerate}
\end{oefening}

\pagebreak
\section{Onbepaalde integraal}

\begin{oefening} {\scriptsize \em IJkingsproef industrieel ingenieur}\\
Bereken
$$\int \dfrac{(1+x)^2}{\sqrt{x}}\;dx$$
\begin{enumerate}[(A)]
\itemsep.5em
  \item $x^{5/2}+x^{3/2}+x^{1/2} + C$
  \item $\dfrac{(x+1)^3}{3\sqrt{x}} + C$
  \item $\dfrac{2}{5}x^{5/2}+\dfrac{2}{3}x^{3/2}+2x^{1/2} + C$
  \item $\dfrac{2}{5}x^{5/2}+\dfrac{4}{3}x^{3/2}+2x^{1/2} + C$
\end{enumerate}
\end{oefening}

\pagebreak
\section{Integratiemethoden}

\pagebreak
\section{Toepassingen op integralen}


%\newpage
\end{document}
