\documentclass[12pt,twoside]{article}

\textwidth 17cm \textheight 25cm \evensidemargin 0cm
\oddsidemargin 0cm \topmargin -2cm
\parindent 0pt
%\parskip \bigskipamount

\usepackage{graphicx}
\usepackage[dutch]{babel}
\usepackage{amssymb,amsthm,amsmath}
%\usepackage{dot2texi}
\usepackage[utf8]{inputenc}
\usepackage{nopageno}
\usepackage{pdfpages}
\usepackage{enumerate}
\usepackage{caption}
\usepackage{wrapfig}
\usepackage{pgf,tikz,pgfplots}
\pgfplotsset{compat=1.15}
\usepackage{color}
\usetikzlibrary{arrows}
\usetikzlibrary{patterns}
\usepackage{fancyhdr}
\pagestyle{fancy}
\usepackage[version=3]{mhchem}
\usepackage{multicol}
\usepackage{fix-cm}
\usepackage{setspace}
\usepackage{mhchem}
\usepackage{xhfill}
\usepackage{parskip}
\usepackage{cancel}
\usepackage{mdframed}
\usepackage{url}
\usepackage{mathtools}
\usepackage{changepage}

\newcommand{\todo}[1]{{\color{red} TODO: #1}}

\newcommand{\degree}{\ensuremath{^\circ}}
\newcommand\rad{\qopname\relax o{\mathrm{rad}}}

\newcommand\ggd{\qopname\relax o{\mathrm{ggd}}}

\pgfmathdeclarefunction{gauss}{2}{%
  \pgfmathparse{1/(#2*sqrt(2*pi))*exp(-((x-#1)^2)/(2*#2^2))}%
}

\def\LRA{\Leftrightarrow}

\newcommand{\zrmbox}{\framebox{\phantom{EXE}}\phantom{X}}
\newcommand{\zrm}[1]{\framebox{#1}}

% environment oefening:
% houdt een teller bij die de oefeningen nummert, probeert ook de oefening op één pagina te houden
\newcounter{noefening}
\setcounter{noefening}{0}
\newenvironment{oefening}
{
  \stepcounter{noefening}
  \pagebreak[0]
  \begin{minipage}{\textwidth}
  \vspace*{0.7cm}{\large\bf Oefening \arabic{noefening}}
}{%
  \end{minipage}
}

\usepackage{calc}

% vraag
\reversemarginpar
\newcounter{punten}
\setcounter{punten}{0}
\newcounter{nvraag}
\setcounter{nvraag}{1}
\newlength{\puntwidth}
\newlength{\boxwidth}
\newcommand{\vraag}[1]{
\settowidth{\puntwidth}{\Large{#1}}
\setlength{\boxwidth}{1.5cm}
\addtolength{\boxwidth}{-\puntwidth}
{\large\bf Vraag \arabic{nvraag} \addtocounter{nvraag}{1}}\vspace*{-0.5cm}
{\marginpar{\color{lightgray}\fbox{\parbox{1.5cm}{\vspace*{1cm}\hspace*{\boxwidth}{\Large{#1}}}}}
\vspace*{0.5cm}}
\addtocounter{punten}{#1}}

% arulefill
\def\arulefill{\leavevmode{\xrfill[-5pt]{0.3pt}[lightgray]\endgraf}\vspace*{0.2cm}}

% \arules{n}
\newcommand{\arules}[1]{
\color{lightgray}
%\vspace*{0.05cm}
\foreach \n in {1,...,#1}{
  \vspace*{0.75cm}
  \hrule height 0.3pt\hfill
}\color{black}\vspace*{0.2cm}}

% \arule{x}
\newcommand{\arule}[1]{
\color{lightgray}{\raisebox{-0.1cm}{\rule[-0.05cm]{#1}{0.3pt}}}\color{black}
}

% \abox{y}
\newcommand{\abox}[1]{
\fbox{
\begin{minipage}{\textwidth- 4\fboxsep}
\hspace*{\textwidth}\vspace{#1}
\end{minipage}
}
}

\newcommand{\ruitjes}[1]{
\definecolor{cqcqcq}{rgb}{0.85,0.85,0.85}
\hspace*{-2.5cm}
\begin{tikzpicture}[scale=1.04,line cap=round,line join=round,>=triangle 45,x=1.0cm,y=1.0cm]
\draw [color=cqcqcq, xstep=0.5cm, ystep=0.5cm] (0,-#1) grid (20.5,0);
\end{tikzpicture}
}


\newcommand{\assenstelsel}[5][1]{
\definecolor{cqcqcq}{rgb}{0.65,0.65,0.65}
\begin{tikzpicture}[line cap=round,line join=round,>=triangle 45,x=#1cm,y=#1cm]
\draw [color=cqcqcq,dash pattern=on 1pt off 1pt, xstep=1.0cm,ystep=1.0cm] (#2,#4) grid (#3,#5);
\draw[->,color=black] (#2,0) -- (#3,0);
%\draw[shift={(1,0)},color=black] (0pt,2pt) -- (0pt,-2pt) node[below] {\footnotesize $1$};
%\draw[color=black] (#3.25,0.07) node [anchor=south west] {$x$};
\draw[->,color=black] (0,#4) -- (0,#5);
%\draw[shift={(0,1)},color=black] (2pt,0pt) -- (-2pt,0pt) node[left] {\footnotesize $1$};
\draw[color=black] (0.09,#5.25) node [anchor=west] {\phantom{$y$}};
%\draw[color=black] (0pt,-10pt) node[right] {\footnotesize $0$};
\end{tikzpicture}
}

\newcommand{\getallenas}[3][1]{
\definecolor{cqcqcq}{rgb}{0.65,0.65,0.65}
\begin{tikzpicture}[scale=#1,line cap=round,line join=round,>=triangle 45,x=1.0cm,y=1.0cm]
\draw [color=cqcqcq,dash pattern=on 1pt off 1pt, xstep=1.0cm,ystep=1.0cm] (#2,-0.2) grid (#3,0.2);
\draw[->,color=black] (#2.25,0) -- (#3.5,0);
\draw[shift={(0,0)},color=black] (0pt,2pt) -- (0pt,-2pt) node[below] {\footnotesize $0$};
\draw[shift={(1,0)},color=black] (0pt,2pt) -- (0pt,-2pt) node[below] {\footnotesize $1$};
\draw[color=black] (#3.25,0.07) node [anchor=south west] {$\mathbb{R}$};
\end{tikzpicture}
}

\newcommand{\visgraad}[1]{\begin{tabular}{p{0.5cm}|p{#1}}&\\\hline\\\end{tabular}}

\newcommand{\tekenschema}[2]{\begin{tabular}{p{0.5cm}|p{#1}}&\\\hline\\[#2]\end{tabular}}

% schema van Horner
\newcommand{\schemahorner}{
\begin{tabular}{p{0.5cm}|p{7cm}}
&\\[1.5cm]
\hline\\
\end{tabular}}

% geef tabular iets meer ruimte
\setlength{\tabcolsep}{14pt}
\renewcommand{\arraystretch}{1.5}

\newcommand{\toets}[3]{
\thispagestyle{plain}
\vspace*{-2.5cm}
\begin{tikzpicture}[remember picture, overlay]
    \node [shift={(15.25 cm,-1.6cm)}] {%
        \includegraphics[width=1.8cm]{/home/ppareit/kaa1415/logokaavelgem.png}%
    };%
\end{tikzpicture}

\begin{tabular}{|llc|c|}
\hline
\vspace*{-0.5cm}
&&&\\
Naam & \arule{4cm} & {\Large\bf KA AVELGEM} & \\
\vspace*{-0.75cm}
&&&\\
Klas & \arule{4cm} & {\Large\bf 20...-...-...} & \\
\hline
\vspace*{-0.75cm}
&&&\\
Toets & {\bf #2} & {\large\bf #1} & Beoordeling\\
\vspace*{-0.75cm}
&&&\\
Onderwerp & \multicolumn{2}{l|}{\bf #3} &\\
\hline
\end{tabular}
}

\newcommand{\oefeningen}[1]{

\fancyhead[LE, RO]{\vspace{0.5cm} #1}
%\thispagestyle{plain}

{\bf \Large \centering Oefeningen: #1}

}

\raggedbottom

\newcommand\vl{\qopname\relax o{\mathrm{vl}}}

\newcommand\dom{\qopname\relax o{\mathrm{dom}}}
\newcommand\ber{\qopname\relax o{\mathrm{ber}}}

\newcommand\mC{\qopname\relax o{\mathrm{mC}}}
\newcommand\uC{\qopname\relax o{\mathrm{{\mu}C}}}
\newcommand\C{\qopname\relax o{\mathrm{C}}}

\newcommand\W{\qopname\relax o{\mathrm{W}}}
\newcommand\kW{\qopname\relax o{\mathrm{kW}}}
\newcommand\kWh{\qopname\relax o{\mathrm{kWh}}}


\newcommand\V{\qopname\relax o{\mathrm{V}}}
\newcommand\ohm{\qopname\relax o{\mathrm{\Omega}}}
\newcommand\kohm{\qopname\relax o{\mathrm{k\Omega}}}


\newcommand\N{\qopname\relax o{\mathrm{N}}}

\newcommand\Nperkg{\qopname\relax o{\mathrm{N/kg}}}

\newcommand\Nperm{\qopname\relax o{\mathrm{N/m}}}

\newcommand\gpermol{\qopname\relax o{\mathrm{g/mol}}}


\newcommand\kgperm{\qopname\relax o{\mathrm{kg/m}}}
\newcommand\kgperdm{\qopname\relax o{\mathrm{kg/dm}}}
\newcommand\gpercm{\qopname\relax o{\mathrm{g/cm}}}
\newcommand\gperml{\qopname\relax o{\mathrm{g/ml}}}


\newcommand{\mA}{\;\mbox{mA}}
\newcommand{\A}{\;\mbox{A}}
\newcommand{\MA}{\;\mbox{MA}}

\newcommand{\us}{\;\mu\mbox{s}}
\newcommand\s{\qopname\relax o{\mathrm{s}}}

\newcommand\h{\qopname\relax o{\mathrm{h}}}

\newcommand{\kmperh}{\;\mbox{km/h}}
\newcommand{\mpers}{\;\mbox{m/s}}
\newcommand{\kmpermin}{\;\mbox{km/min}}
\newcommand{\kmpers}{\;\mbox{km/s}}

\newcommand{\mph}{\;\mbox{mph}}

\newcommand{\Hz}{\;\mbox{Hz}}

\newcommand\Gm{\qopname\relax o{\mathrm{Gm}}}
\newcommand\Mm{\qopname\relax o{\mathrm{Mm}}}
\newcommand\km{\qopname\relax o{\mathrm{km}}}
\newcommand\hm{\qopname\relax o{\mathrm{hm}}}
\newcommand\dam{\qopname\relax o{\mathrm{dam}}}
\newcommand\m{\qopname\relax o{\mathrm{m}}}
\newcommand\dm{\qopname\relax o{\mathrm{dm}}}
\newcommand\cm{\qopname\relax o{\mathrm{cm}}}
\newcommand\mm{\qopname\relax o{\mathrm{mm}}}
\newcommand\um{\qopname\relax o{\mathrm{{\mu}m}}}
\newcommand\nm{\qopname\relax o{\mathrm{nm}}}


\newcommand\Gg{\qopname\relax o{\mathrm{Gg}}}
\newcommand\Mg{\qopname\relax o{\mathrm{Mg}}}
\newcommand\kg{\qopname\relax o{\mathrm{kg}}}
\newcommand\hg{\qopname\relax o{\mathrm{hg}}}
\renewcommand\dag{\qopname\relax o{\mathrm{dag}}}
\newcommand\g{\qopname\relax o{\mathrm{g}}}
\newcommand\dg{\qopname\relax o{\mathrm{dg}}}
\newcommand\cg{\qopname\relax o{\mathrm{cg}}}
\newcommand\mg{\qopname\relax o{\mathrm{mg}}}
\newcommand\ug{\qopname\relax o{\mathrm{{\mu}g}}}
\renewcommand\ng{\qopname\relax o{\mathrm{ng}}}

\newcommand\ton{\qopname\relax o{\mathrm{ton}}}

\newcommand\Gl{\qopname\relax o{\mathrm{Gl}}}
\newcommand\Ml{\qopname\relax o{\mathrm{Ml}}}
\newcommand\kl{\qopname\relax o{\mathrm{kl}}}
\newcommand\hl{\qopname\relax o{\mathrm{hl}}}
\newcommand\dal{\qopname\relax o{\mathrm{dal}}}
\renewcommand\l{\qopname\relax o{\mathrm{l}}}
\newcommand\dl{\qopname\relax o{\mathrm{dl}}}
\newcommand\cl{\qopname\relax o{\mathrm{cl}}}
\newcommand\ml{\qopname\relax o{\mathrm{ml}}}
\newcommand\ul{\qopname\relax o{\mathrm{{\mu}l}}}
\newcommand\nl{\qopname\relax o{\mathrm{nl}}}

\newcommand\MJ{\qopname\relax o{\mathrm{MJ}}}
\newcommand\kJ{\qopname\relax o{\mathrm{kJ}}}
\newcommand\J{\qopname\relax o{\mathrm{J}}}

\newcommand\T{\qopname\relax o{\mathrm{T}}}
\newcommand\uT{\qopname\relax o{\mathrm{{\mu}T}}}

\newcommand\grC{\qopname\relax o{\mathrm{{\degree}C}}}

\newcommand\K{\qopname\relax o{\mathrm{K}}}
\newcommand\calperK{\qopname\relax o{\mathrm{cal/K}}}

\newcommand\hPa{\qopname\relax o{\mathrm{hPa}}}
\newcommand\Pa{\qopname\relax o{\mathrm{Pa}}}

\newcommand\dB{\qopname\relax o{\mathrm{dB}}}

\newcommand\Var{\qopname\relax o{\mathrm{Var}}}

\newcommand{\EE}[1]{\cdot 10^{#1}}

\onehalfspacing

%\setlength{\headsep}{0cm}

\newenvironment{exlist}[1] %
{ \begin{multicols}{#1}
  \begin{enumerate}[(a)]
    \setlength{\itemsep}{0.5em} }
{ \end{enumerate}
  \end{multicols} }




\usepackage{versions}
\includeversion{theorie}
%\excludeversion{theorie}

\begin{document}

\pagestyle{fancy}
\lhead{}
\rhead{Oefeningen Irrationale Functies}

\begin{theorie}

  \thispagestyle{empty}
  \begin{center}
    \begin{mdframed}
      \centering
      \fontsize{40}{60}\selectfont Irrationale Functies
    \end{mdframed}
    \vfill
    \definecolor{aqaqaq}{rgb}{0.63,0.63,0.63}
    \definecolor{wqwqwq}{rgb}{0.38,0.38,0.38}
    \begin{tikzpicture}[scale=1.5, line cap=round,line join=round,>=triangle 45,x=1.0cm,y=1.0cm]
      \draw[->,color=black] (-3.43,0) -- (3.78,0);
      \foreach \x in {-3,-2,-1,1,2,3}
      \draw[shift={(\x,0)},color=black] (0pt,2pt) -- (0pt,-2pt) node[below] {\footnotesize $\x$};
      \draw[->,color=black] (0,-3.46) -- (0,3.9);
      \foreach \y in {-3,-2,-1,1,2,3}
      \draw[shift={(0,\y)},color=black] (2pt,0pt) -- (-2pt,0pt) node[left] {\footnotesize $\y$};
      \draw[color=black] (0pt,-10pt) node[right] {\footnotesize $0$};
      \clip(-3.43,-3.46) rectangle (3.78,3.9);
      \draw[line width=1.6pt,color=wqwqwq, smooth,samples=100,domain=-1.9999995090909108:1.999994949917767] plot(\x,{sqrt(4-(\x)^2)});
      \draw[line width=1.6pt,color=aqaqaq, smooth,samples=100,domain=-1.9999995090909108:1.999994949917767] plot(\x,{0-sqrt(4-(\x)^2)});
      \draw (0.58,2.43) node[anchor=north west] {$f: y=\sqrt{4-x^2}$};
      \draw (1.46,-1.5) node[anchor=north west] {$g: y=-\sqrt{4-x^2}$};
    \end{tikzpicture}
    \vfill
  \end{center}

  \subsection*{Doelstellingen}
  \vspace*{-0.8cm}
  {\singlespacing
    Je \hfill  {\scriptsize(LP2006-059, LI1.9, ET32,14,31)}
    \begin{itemize}
      \itemsep-0.2em
    \item kan irrationale vergelijkingen oplossen
    \item kan aan de hand van het functievoorschrift een tabel, het domein, de nulwaarden en het tekenverloop bepalen van irrationale functies
    \item kan aan de hand van de grafiek domein, bereik, nulwaarden, tekenverloop, stijgen\&dalen, extrema en asymptotisch gedrag bepalen van irrationale functies
    \item kan vraagstukken/problemen oplossen die aanleiding ge-
      ven tot een irrationale vergelijking, ongelijkheid of func-
      tie, eventueel met behulp van ICT
    \item kan extremumvraagstukken (ook van buiten de wiskunde)
      die aanleiding geven tot irrationale functies, oplossen
    \end{itemize}}

  \thispagestyle{empty}
  \mbox{}
  \newpage
  \clearpage
  \thispagestyle{empty}
  % \mbox{}
  \tableofcontents
  \newpage
  \clearpage
  \pagenumbering{arabic} 

  \fancyhead[RO,LE]{Irrationale functies}
  \fancyhead[RE,LO]{}

\end{theorie}

\section{Irrationale vergelijkingen}

\begin{theorie}
\subsection{Definitie}

\begin{mdframed}
Een {\bf irrationale vergelijking} is een vergelijking waarin de onbekende $x$ voorkomt onder het wortelteken.
\end{mdframed}

\subsection{Voorbeelden}
\begin{enumerate}[(a)]
  \item $\sqrt{2x+3}=7$
  \item $\sqrt{x^2-2x+5}=13$
  \item $\sqrt{x^2-4}=\sqrt{x+2}$
\end{enumerate}
  
  
We zullen vooral irrationale vergelijkingen van de volgende simpele vorm oplossen:
$$\sqrt{ax^2+bx+c}=d\qquad\mbox{ met } d\in\mathbb{R}^+$$

\subsection{Algebraïsche oplossingsmethode}

\paragraph*{Voorbeelden}
\begin{enumerate}[(a)]
  \item Los $\sqrt{2x+3}=7$ op in $\mathbb{R}$:\\
  Voordat we de vergelijking effectief oplossen, zullen we controleren of de oplossing überhaupt kan bestaan. Omdat er in de vergelijking een vierkantswortel staat, moet alles onder de wortel positief zijn. We eisen dus als{\bf bestaansvoorwaarde (BV)}:
  \begin{align*}
         && 2x+3 &\geq 0 \\
    \LRA && x &\geq -\dfrac{3}{2}
  \end{align*}
  \begin{align*}
         && \sqrt{2x+3} &= 7 \\
    \LRA && \left(\sqrt{2x+3}\right)^2 &= \left(7\right)^2\\
    \LRA && 2x+3 &= 49\\
    \LRA && x &= 23\\
  \end{align*}
  Merk op dat $23\geq -\dfrac{3}{2}$ en er dus voldaan is aan de BV. Dus 
  $$V=\{23\}$$
  \item Los $\sqrt{x^2-4x+8}=2$ op in $\mathbb{R}$:\\
  Als BV eisen we $x^2-4x+8\geq 0$, we hebben gezien (Hoofdstuk Veeltermfuncties) dat we deze ongelijkheid best oplossen m.b.v. een tekenverloop.\\
  {\bf Nulwaarden functie onder de wortel:}
  \begin{align*}
         && x^2-4x+8 &= 0\\
         && D &= 16-4\cdot 1\cdot 8\\
         &&   &= 16-32\\
         &&   &= -16 < 0\\
         && &\Rightarrow \text{geen nulwaarden}
  \end{align*}
  {\bf Tekenverloop functie onder de wortel:}
  \begin{center}
    \begin{tabular}{c|ccc}
    $x$ & & & \\
    \hline
    $y$ & & + &  
    \end{tabular}
  \end{center}
  {\bf Bestaansvoorwaarde (BV):}
  $$x\in\mathbb{R}$$
  {\bf Oplossen irrationale vergelijking:}
  \begin{align*}
         && \sqrt{x^2-4x+8} &= 2 \\
    \LRA && \left(\sqrt{x^2-4x+8}\right)^2 &= \left(2\right)^2\\
    \LRA && x^2-4x+8 &= 4\\
    \LRA && x^2-4x+4 &= 0\\
    \LRA && (x-2)^2 &= 0\\
    \LRA && x &= 2\\
  \end{align*}
  {\bf Controleren oplossing m.b.v. bestaansvoorwaarde:}\\
  $$2\in\mathbb{R}$$\\
  {\bf Noteren van de oplossingenverzameling:}\\
  $$V=\{2\}$$
\end{enumerate}

\end{theorie}

\begin{oefening}
Los de volgende irrationale vergelijkingen op in $\mathbb{R}$:
\begin{multicols}{2}
\begin{enumerate}[(a)]
  \item $\sqrt{x+2}=2$
  \item $\sqrt{3-x}=2$
  \item $\sqrt{7-3x}=1$
  \item $\sqrt{x^2+x+3}=3$
  \item $\sqrt{x^2-6x+13}=2$
  \item $\sqrt{25-x^2}=4$
  \item $6-\sqrt{8+x}=0$
  \item $\sqrt{x-1}-8=0$
  \item $-\sqrt{49-x^2}+7=0$
  \item $\sqrt{11x^2+2x-9}-3=0$
  \item $12-\sqrt{4x+9x^2}=0$
  \item $\sqrt{4x^2-2}=2$
\end{enumerate}
\end{multicols}
\end{oefening}

\begin{oefening}* Toon aan dat de oplossingen van een irrationale vergelijking van de vorm $\sqrt{ax+b}=c>0$ altijd voldoen aan de bestaansvoorwaarde $ax+b\geq0$.
\end{oefening}


\begin{oefening}*
De volgende irrationale vergelijking
$$\sqrt{x+8}=x+2$$
staat niet in onze eenvoudige standaardvorm ($\sqrt{ax^2+bx+c}=d\geq0$).
\begin{enumerate}[(a)]
  \item Wat is het probleem en is dit probleem oplosbaar?
  \item Kan je deze irrationale vergelijking toch oplossen?
  \item Maak de proef van je mogelijke oplossingen, wat wil dit zeggen voor de oplossingenverzameling?
  \item Voordat je de vierkantswortel kan wegwerken, moet je een voorwaarde stellen op het rechterlid, welke voorwaarde is dit?
\end{enumerate}
{\em Opmerking: Deze voorwaarde wordt de kwadrateringsvoorwaarde (KV) genoemd.}
\end{oefening}

\begin{oefening}*
Los de volgende irrationale vergelijkingen op in $\mathbb{R}$:
\begin{multicols}{2}
\begin{enumerate}[(a)]
  \item $x-2=\sqrt{7-2x}$
  \item $x-1=\sqrt{x+7}+4$
  \item $2\sqrt{2x^2-3x-2}=x-2$
  \item $\sqrt{x-4}+\sqrt{x+1}=5$
  \item $\sqrt{3x-5}=1+\sqrt{x+3}$
  \item $\sqrt{4x+1}=\sqrt{9x^2-1}$
\end{enumerate}
\end{multicols}
\end{oefening}

\begin{oefening}* % pauls online notes
Los de volgende irrationale vergelijkingen op in $\mathbb{R}$:
\begin{multicols}{2}
\begin{enumerate}[(a)]
  \item $x=\sqrt{x+6}$
  \item $y+\sqrt{y-4}=4$
  \item $1=t+\sqrt{2t-3}$
  \item $\sqrt{5z+6}-2=z$
  \item $\sqrt{2x-1}-\sqrt{x-4}=2$
  \item $\sqrt{t+7}+2=\sqrt{3-t}$
\end{enumerate}
\end{multicols}
\end{oefening}

\pagebreak
\section{Bespreken van irrationale functies}

\begin{oefening}
Welke deeltaken zullen we uitvoeren wanneer we een irrationale functie bespreken?
\end{oefening}

\begin{oefening}
Als je een functiewaardentabel opstelt, hoe weet je dan van waar tot waar je de $x$-waarden moet nemen?
\end{oefening}

\begin{oefening}
Je nulwaarden kan je aflezen van de grafiek. Soms kan het echter handig zijn om ze algebraïsch te bepalen. Hoe? Waarom?
\end{oefening}

\begin{oefening}
Bespreek volgende irrationale functies:
\begin{enumerate}[(a)]
  \item $f(x)=\sqrt{x-1}$
  \item $f(x)=\sqrt{4+x^2}$
  \item $f(x)=\sqrt{4-x^2}$
  \item $f(x)=\sqrt{x^2-4}$
  \item $f(x)=\sqrt{x^2-6x+13}-2$
  \item $f(x)=2\sqrt{9-x^2}$
  \item $f(x)=\sqrt{25-x^2}-5$
  \item $f(x)=\sqrt{x^2+3x-2}$
  \item $f(x)=\sqrt{-x^2+x+6}$
\end{enumerate}
\end{oefening}

\begin{oefening}
Gebruik geogebra om je resultaten uit de vorige oefening te controleren.
\end{oefening}

\begin{oefening}*
De cirkel met middelpunt de oorsprong en met straal 2 wordt gegeven door de vergelijking
$$x^2+y^2=4\;.$$
Vorm deze vergelijking om tot je twee irrationale functies hebt, één die boven de $x$-as ligt en één die onder de $x$-as ligt.
\end{oefening}

\pagebreak
\section{Toepassingen}

\begin{oefening} %Jennekens, Beknopte analyse, p220
Een satelliet die op $x$ km afstand van het centrum van de aarde om de aarde draait, kan ontsnappen aan de aantrekkingskracht van de aarde bij een snelheid $v$ gegeven door:
$$v=\dfrac{893}{\sqrt{x}}\qquad\qquad\mbox{ (in km/s)}$$
Hoe hoog boven de oppervlakte van de aarde draait een satelliet als zijn ontsnappingssnelheid $11$ km/s is? Neem als straal van de aarde 6378 km.
\end{oefening}

\begin{oefening} %Jennekens, Beknopte analyse, p220
Wanneer is de positieve vierkantswortel van een reëel getal groter dan dat getal?
\end{oefening}

\begin{oefening} %Jennekens, Beknopte analyse, p220
Wanneer is de positieve vierkantswortel van een getal groter dan het kwadraat van dat getal?
\end{oefening}

\begin{oefening} %Jennekens, Beknopte analyse, p220
Wanneer is de positieve vierkantswortel van een reëel getal groter dan het omgekeerde van dat getal?
\end{oefening}


\begin{oefening} %Jennekens, Beknopte analyse, p221
Voor een $\Delta ABC$ geldt:
$$a=21\qquad b=20\qquad c=13$$
Bereken de hoogte uit $A$ door de stelling van Pythagoras tweemaal te gebruiken.
\end{oefening}

\begin{oefening} %Jennekens, Beknopte analyse, p221
De periode $T$ in seconden van een slinger met een lengte van $x$ meter wordt gegeven door de formule:
$$T=2\pi\sqrt{\dfrac{x}{g}}\;.$$
Daarin is $g$ de gravitatieconstante, gelijk aan $9.8$ m/s$^s$.
\begin{enumerate}[(a)]
  \item Maak de grafiek van de functie die $x$ op $T$ afbeeldt.
  \item Hoe lang moet de slinger van een klok zijn om een periode van 4 s te verkrijgen?
\end{enumerate}
\end{oefening}

\begin{oefening} %Pareit-Piaret-Piraet-Pirate
Op het pretpark staat een toffe attractie, het {\em Piraten Schip}. De bezoekers zitten in een groot schip. Het schip schommelt van voor naar achter, net zoals een slinger. Het duurt tussen de 7 en 8 seconden om éénmaal van voor naar achteren en terug te schommelen. De functie die de tijd $t$ in seconden geeft voor volledige zwaai, gebaseerd op de hoogte van de arm van de schommel $h$ in meter is
$$t=2\pi\sqrt{\dfrac{h}{10}}\;.$$
Hoe hoog moet de arm dan ongeveer zijn om één slingerbeweging uit te voeren? Los dit op met behulp van een grafiek.
\end{oefening}

\begin{oefening} % Mathematics in Everyday Things
Als je naar de zee kijkt, dan is de afstand $d$ in kilometer die je kan zien tot aan de horizon gegeven door
$$d=3.86\sqrt{h}$$
waarbij $h$ je hoogte in meter boven het water is. Geef de grafiek van deze functie. Bepaal de hoogte waarop je 10 kilometer ver ziet.
\end{oefening}

\begin{oefening} % http://www.math30.ca/lessons/polynomials/radicalFunctions/radicalFunctions_notes.php
Een ladder is 3 meter lang en leunt tegen een muur. De basis van de ladder staat op $d$ meter van de muur en de top van de ladder is $h$ meter boven de grond.
\begin{enumerate}[(a)]
  \item Geef een functie $h(d)$ die de hoogte van de ladder geeft in functie van zijn afstand basis-muur $d$.
  \item Geef de grafiek, het domein en het bereik van de functie. Welke oriëntatie heeft de ladder als $d=0$ en als $d=3$.
  \item Hoe ver is de basis van de ladder verwijderd van de muur als de top van de ladder $\sqrt{5}$ meter boven de grond is?
\end{enumerate}
\end{oefening}

\begin{oefening} % http://www.math30.ca/lessons/polynomials/radicalFunctions/radicalFunctions_notes.php
We laten een bal van $h$ meter hoog vallen. Het duurt $t=\sqrt{\dfrac{h}{4.9}}\s$ voordat de bal de grond raakt.
\begin{enumerate}[(a)]
  \item Als we de bal van dubbel zo hoog laten vallen, welke invloed zal dit hebben op de tijd voordat de bal de grond raakt?
  \item Als we de bal van één-vierde zo hoog laten vallen, welke invloed zal dit hebben op de tijd voordat de bal de grond raakt?
  \item Stel dat we initieel de bal van $h=4\m$ lieten vallen. Maak een grafiek waarbij we de tijd uitzetten ten opzichte van hoogte en zorg dat alle relevante experimenten zichtbaar zijn op de grafiek.
\end{enumerate}
\end{oefening}

\begin{oefening} % http://www.math30.ca/lessons/polynomials/radicalFunctions/radicalFunctions_notes.php
Een wegwerpbekertje heeft de vorm van een kegel. Het volume van de kegel is $V (\cm^3)$, de radius is $r (\cm)$, de hoogte is $h (\cm)$ en de schuine hoogte is $5 \cm$. Geef de functie $V(r)$ die het volume van het bekertje geeft in functie van $r$. Merk op dat het volume van een kegel gegeven wordt door $V=\dfrac{1}{3}\pi r^2 h$.
\end{oefening}



%%%%%%%%%%%%%%%%%%%%%%%%%%%%%%%%%%%%%%%%%%%%%%%%%%%%%%%%%%%%%%%%%%%%%% 
\end{document}




\begin{minipage}[c]{0.4\textwidth}
\end{minipage}
\begin{minipage}[c]{0.6\textwidth}
  \dotlines{10}
\end{minipage}
















