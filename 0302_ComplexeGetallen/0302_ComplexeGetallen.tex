\documentclass[12pt,twoside]{article}

\textwidth 17cm \textheight 25cm \evensidemargin 0cm
\oddsidemargin 0cm \topmargin -2cm
\parindent 0pt
%\parskip \bigskipamount

\usepackage{graphicx}
\usepackage[dutch]{babel}
\usepackage{amssymb,amsthm,amsmath}
%\usepackage{dot2texi}
\usepackage[utf8]{inputenc}
\usepackage{nopageno}
\usepackage{pdfpages}
\usepackage{enumerate}
\usepackage{caption}
\usepackage{wrapfig}
\usepackage{pgf,tikz,pgfplots}
\pgfplotsset{compat=1.15}
\usepackage{color}
\usetikzlibrary{arrows}
\usetikzlibrary{patterns}
\usepackage{fancyhdr}
\pagestyle{fancy}
\usepackage[version=3]{mhchem}
\usepackage{multicol}
\usepackage{fix-cm}
\usepackage{setspace}
\usepackage{mhchem}
\usepackage{xhfill}
\usepackage{parskip}
\usepackage{cancel}
\usepackage{mdframed}
\usepackage{url}
\usepackage{mathtools}
\usepackage{changepage}

\newcommand{\todo}[1]{{\color{red} TODO: #1}}

\newcommand{\degree}{\ensuremath{^\circ}}
\newcommand\rad{\qopname\relax o{\mathrm{rad}}}

\newcommand\ggd{\qopname\relax o{\mathrm{ggd}}}

\pgfmathdeclarefunction{gauss}{2}{%
  \pgfmathparse{1/(#2*sqrt(2*pi))*exp(-((x-#1)^2)/(2*#2^2))}%
}

\def\LRA{\Leftrightarrow}

\newcommand{\zrmbox}{\framebox{\phantom{EXE}}\phantom{X}}
\newcommand{\zrm}[1]{\framebox{#1}}

% environment oefening:
% houdt een teller bij die de oefeningen nummert, probeert ook de oefening op één pagina te houden
\newcounter{noefening}
\setcounter{noefening}{0}
\newenvironment{oefening}
{
  \stepcounter{noefening}
  \pagebreak[0]
  \begin{minipage}{\textwidth}
  \vspace*{0.7cm}{\large\bf Oefening \arabic{noefening}}
}{%
  \end{minipage}
}

\usepackage{calc}

% vraag
\reversemarginpar
\newcounter{punten}
\setcounter{punten}{0}
\newcounter{nvraag}
\setcounter{nvraag}{1}
\newlength{\puntwidth}
\newlength{\boxwidth}
\newcommand{\vraag}[1]{
\settowidth{\puntwidth}{\Large{#1}}
\setlength{\boxwidth}{1.5cm}
\addtolength{\boxwidth}{-\puntwidth}
{\large\bf Vraag \arabic{nvraag} \addtocounter{nvraag}{1}}\vspace*{-0.5cm}
{\marginpar{\color{lightgray}\fbox{\parbox{1.5cm}{\vspace*{1cm}\hspace*{\boxwidth}{\Large{#1}}}}}
\vspace*{0.5cm}}
\addtocounter{punten}{#1}}

% arulefill
\def\arulefill{\leavevmode{\xrfill[-5pt]{0.3pt}[lightgray]\endgraf}\vspace*{0.2cm}}

% \arules{n}
\newcommand{\arules}[1]{
\color{lightgray}
%\vspace*{0.05cm}
\foreach \n in {1,...,#1}{
  \vspace*{0.75cm}
  \hrule height 0.3pt\hfill
}\color{black}\vspace*{0.2cm}}

% \arule{x}
\newcommand{\arule}[1]{
\color{lightgray}{\raisebox{-0.1cm}{\rule[-0.05cm]{#1}{0.3pt}}}\color{black}
}

% \abox{y}
\newcommand{\abox}[1]{
\fbox{
\begin{minipage}{\textwidth- 4\fboxsep}
\hspace*{\textwidth}\vspace{#1}
\end{minipage}
}
}

\newcommand{\ruitjes}[1]{
\definecolor{cqcqcq}{rgb}{0.85,0.85,0.85}
\hspace*{-2.5cm}
\begin{tikzpicture}[scale=1.04,line cap=round,line join=round,>=triangle 45,x=1.0cm,y=1.0cm]
\draw [color=cqcqcq, xstep=0.5cm, ystep=0.5cm] (0,-#1) grid (20.5,0);
\end{tikzpicture}
}


\newcommand{\assenstelsel}[5][1]{
\definecolor{cqcqcq}{rgb}{0.65,0.65,0.65}
\begin{tikzpicture}[line cap=round,line join=round,>=triangle 45,x=#1cm,y=#1cm]
\draw [color=cqcqcq,dash pattern=on 1pt off 1pt, xstep=1.0cm,ystep=1.0cm] (#2,#4) grid (#3,#5);
\draw[->,color=black] (#2,0) -- (#3,0);
%\draw[shift={(1,0)},color=black] (0pt,2pt) -- (0pt,-2pt) node[below] {\footnotesize $1$};
%\draw[color=black] (#3.25,0.07) node [anchor=south west] {$x$};
\draw[->,color=black] (0,#4) -- (0,#5);
%\draw[shift={(0,1)},color=black] (2pt,0pt) -- (-2pt,0pt) node[left] {\footnotesize $1$};
\draw[color=black] (0.09,#5.25) node [anchor=west] {\phantom{$y$}};
%\draw[color=black] (0pt,-10pt) node[right] {\footnotesize $0$};
\end{tikzpicture}
}

\newcommand{\getallenas}[3][1]{
\definecolor{cqcqcq}{rgb}{0.65,0.65,0.65}
\begin{tikzpicture}[scale=#1,line cap=round,line join=round,>=triangle 45,x=1.0cm,y=1.0cm]
\draw [color=cqcqcq,dash pattern=on 1pt off 1pt, xstep=1.0cm,ystep=1.0cm] (#2,-0.2) grid (#3,0.2);
\draw[->,color=black] (#2.25,0) -- (#3.5,0);
\draw[shift={(0,0)},color=black] (0pt,2pt) -- (0pt,-2pt) node[below] {\footnotesize $0$};
\draw[shift={(1,0)},color=black] (0pt,2pt) -- (0pt,-2pt) node[below] {\footnotesize $1$};
\draw[color=black] (#3.25,0.07) node [anchor=south west] {$\mathbb{R}$};
\end{tikzpicture}
}

\newcommand{\visgraad}[1]{\begin{tabular}{p{0.5cm}|p{#1}}&\\\hline\\\end{tabular}}

\newcommand{\tekenschema}[2]{\begin{tabular}{p{0.5cm}|p{#1}}&\\\hline\\[#2]\end{tabular}}

% schema van Horner
\newcommand{\schemahorner}{
\begin{tabular}{p{0.5cm}|p{7cm}}
&\\[1.5cm]
\hline\\
\end{tabular}}

% geef tabular iets meer ruimte
\setlength{\tabcolsep}{14pt}
\renewcommand{\arraystretch}{1.5}

\newcommand{\toets}[3]{
\thispagestyle{plain}
\vspace*{-2.5cm}
\begin{tikzpicture}[remember picture, overlay]
    \node [shift={(15.25 cm,-1.6cm)}] {%
        \includegraphics[width=1.8cm]{/home/ppareit/kaa1415/logokaavelgem.png}%
    };%
\end{tikzpicture}

\begin{tabular}{|llc|c|}
\hline
\vspace*{-0.5cm}
&&&\\
Naam & \arule{4cm} & {\Large\bf KA AVELGEM} & \\
\vspace*{-0.75cm}
&&&\\
Klas & \arule{4cm} & {\Large\bf 20...-...-...} & \\
\hline
\vspace*{-0.75cm}
&&&\\
Toets & {\bf #2} & {\large\bf #1} & Beoordeling\\
\vspace*{-0.75cm}
&&&\\
Onderwerp & \multicolumn{2}{l|}{\bf #3} &\\
\hline
\end{tabular}
}

\newcommand{\oefeningen}[1]{

\fancyhead[LE, RO]{\vspace{0.5cm} #1}
%\thispagestyle{plain}

{\bf \Large \centering Oefeningen: #1}

}

\raggedbottom

\newcommand\vl{\qopname\relax o{\mathrm{vl}}}

\newcommand\dom{\qopname\relax o{\mathrm{dom}}}
\newcommand\ber{\qopname\relax o{\mathrm{ber}}}

\newcommand\mC{\qopname\relax o{\mathrm{mC}}}
\newcommand\uC{\qopname\relax o{\mathrm{{\mu}C}}}
\newcommand\C{\qopname\relax o{\mathrm{C}}}

\newcommand\W{\qopname\relax o{\mathrm{W}}}
\newcommand\kW{\qopname\relax o{\mathrm{kW}}}
\newcommand\kWh{\qopname\relax o{\mathrm{kWh}}}


\newcommand\V{\qopname\relax o{\mathrm{V}}}
\newcommand\ohm{\qopname\relax o{\mathrm{\Omega}}}
\newcommand\kohm{\qopname\relax o{\mathrm{k\Omega}}}


\newcommand\N{\qopname\relax o{\mathrm{N}}}

\newcommand\Nperkg{\qopname\relax o{\mathrm{N/kg}}}

\newcommand\Nperm{\qopname\relax o{\mathrm{N/m}}}

\newcommand\gpermol{\qopname\relax o{\mathrm{g/mol}}}


\newcommand\kgperm{\qopname\relax o{\mathrm{kg/m}}}
\newcommand\kgperdm{\qopname\relax o{\mathrm{kg/dm}}}
\newcommand\gpercm{\qopname\relax o{\mathrm{g/cm}}}
\newcommand\gperml{\qopname\relax o{\mathrm{g/ml}}}


\newcommand{\mA}{\;\mbox{mA}}
\newcommand{\A}{\;\mbox{A}}
\newcommand{\MA}{\;\mbox{MA}}

\newcommand{\us}{\;\mu\mbox{s}}
\newcommand\s{\qopname\relax o{\mathrm{s}}}

\newcommand\h{\qopname\relax o{\mathrm{h}}}

\newcommand{\kmperh}{\;\mbox{km/h}}
\newcommand{\mpers}{\;\mbox{m/s}}
\newcommand{\kmpermin}{\;\mbox{km/min}}
\newcommand{\kmpers}{\;\mbox{km/s}}

\newcommand{\mph}{\;\mbox{mph}}

\newcommand{\Hz}{\;\mbox{Hz}}

\newcommand\Gm{\qopname\relax o{\mathrm{Gm}}}
\newcommand\Mm{\qopname\relax o{\mathrm{Mm}}}
\newcommand\km{\qopname\relax o{\mathrm{km}}}
\newcommand\hm{\qopname\relax o{\mathrm{hm}}}
\newcommand\dam{\qopname\relax o{\mathrm{dam}}}
\newcommand\m{\qopname\relax o{\mathrm{m}}}
\newcommand\dm{\qopname\relax o{\mathrm{dm}}}
\newcommand\cm{\qopname\relax o{\mathrm{cm}}}
\newcommand\mm{\qopname\relax o{\mathrm{mm}}}
\newcommand\um{\qopname\relax o{\mathrm{{\mu}m}}}
\newcommand\nm{\qopname\relax o{\mathrm{nm}}}


\newcommand\Gg{\qopname\relax o{\mathrm{Gg}}}
\newcommand\Mg{\qopname\relax o{\mathrm{Mg}}}
\newcommand\kg{\qopname\relax o{\mathrm{kg}}}
\newcommand\hg{\qopname\relax o{\mathrm{hg}}}
\renewcommand\dag{\qopname\relax o{\mathrm{dag}}}
\newcommand\g{\qopname\relax o{\mathrm{g}}}
\newcommand\dg{\qopname\relax o{\mathrm{dg}}}
\newcommand\cg{\qopname\relax o{\mathrm{cg}}}
\newcommand\mg{\qopname\relax o{\mathrm{mg}}}
\newcommand\ug{\qopname\relax o{\mathrm{{\mu}g}}}
\renewcommand\ng{\qopname\relax o{\mathrm{ng}}}

\newcommand\ton{\qopname\relax o{\mathrm{ton}}}

\newcommand\Gl{\qopname\relax o{\mathrm{Gl}}}
\newcommand\Ml{\qopname\relax o{\mathrm{Ml}}}
\newcommand\kl{\qopname\relax o{\mathrm{kl}}}
\newcommand\hl{\qopname\relax o{\mathrm{hl}}}
\newcommand\dal{\qopname\relax o{\mathrm{dal}}}
\renewcommand\l{\qopname\relax o{\mathrm{l}}}
\newcommand\dl{\qopname\relax o{\mathrm{dl}}}
\newcommand\cl{\qopname\relax o{\mathrm{cl}}}
\newcommand\ml{\qopname\relax o{\mathrm{ml}}}
\newcommand\ul{\qopname\relax o{\mathrm{{\mu}l}}}
\newcommand\nl{\qopname\relax o{\mathrm{nl}}}

\newcommand\MJ{\qopname\relax o{\mathrm{MJ}}}
\newcommand\kJ{\qopname\relax o{\mathrm{kJ}}}
\newcommand\J{\qopname\relax o{\mathrm{J}}}

\newcommand\T{\qopname\relax o{\mathrm{T}}}
\newcommand\uT{\qopname\relax o{\mathrm{{\mu}T}}}

\newcommand\grC{\qopname\relax o{\mathrm{{\degree}C}}}

\newcommand\K{\qopname\relax o{\mathrm{K}}}
\newcommand\calperK{\qopname\relax o{\mathrm{cal/K}}}

\newcommand\hPa{\qopname\relax o{\mathrm{hPa}}}
\newcommand\Pa{\qopname\relax o{\mathrm{Pa}}}

\newcommand\dB{\qopname\relax o{\mathrm{dB}}}

\newcommand\Var{\qopname\relax o{\mathrm{Var}}}

\newcommand{\EE}[1]{\cdot 10^{#1}}

\onehalfspacing

%\setlength{\headsep}{0cm}

\newenvironment{exlist}[1] %
{ \begin{multicols}{#1}
  \begin{enumerate}[(a)]
    \setlength{\itemsep}{0.5em} }
{ \end{enumerate}
  \end{multicols} }




\usepackage{versions}
\excludeversion{theorie}
% \includeversion{theorie}

\begin{document}

\pagestyle{fancy}
\lhead{}
\rhead{Oefeningen Complexe Getallen}

\begin{theorie}

  \thispagestyle{empty}
  \begin{center}
    \begin{mdframed}
      \centering
      \fontsize{40}{60}\selectfont Goniometrische Getallen
    \end{mdframed}
    \vfill
    \vfill
  \end{center}

  \subsection*{Doelstellingen}
  \vspace*{-0.8cm}
  {\singlespacing
    Je \hfill  {\scriptsize(LP2006-059, LI1.9, ET32,14,31)}
    \begin{itemize}
      \itemsep-0.2em
    \item kent het begrip complex getal
    \item kan complexe getallen optellen, aftrekken, vermenigvuldigen en delen
    \item kan de $k$-de macht van een complex getal berekenen
    \item kan algebraïsch vierkantswortels uit een complex getal berekenen
    \item kan vergelijkingen van de tweede graad met reële coëfficiënten oplossen in $\mathbb{C}$
    \item kan complexe getallen voorstellen in het vlak van Gauss
    \item kent de goniometrische gedaante van een complex getal
    \item kan product, quotiënt en macht berekenen van complexe
      getallen in goniometrische gedaante
    \item kent de formule van de Moivre
    \item kan goniometrisch de $n$-de wortels berekenen uit een complex getal in goniometrische gedaante
    \end{itemize}}

  \thispagestyle{empty}
  \mbox{}
  \newpage
  \clearpage
  \thispagestyle{empty}
  % \mbox{}
  \tableofcontents
  \newpage
  \clearpage
  \pagenumbering{arabic}

  \fancyhead[RO,LE]{Complexe Getallen}
  \fancyhead[RE,LO]{}

\end{theorie}

\section{Definitie}

\begin{oefening}
  Maak gebruik van een venndiagram om de volgende getallen in de juiste getalverzameling te plaatsten:
  $$-3\qquad\pi\qquad\dfrac{3}{4}\qquad2+3i\qquad10^{10}\qquad-\sqrt{9}$$
\end{oefening}

\begin{oefening}
  Als $i^2=-1$, aan wat is dan $i^4$ gelijk?
\end{oefening}

\begin{oefening}
  Bepaal van de volgende getallen het reëel en het imaginair deel:
  $$2+3i\qquad -i+1\qquad 0 \qquad 2i$$
\end{oefening}

\pagebreak
\section{Basisbewerkingen met complexe getallen}

\begin{oefening}
  Gegeven het complex getal $a+bi$, bepaal het neutraal element en het symmetrisch element in $\mathbb{C}, +$.
\end{oefening}

\begin{oefening}*
  Gegeven het complex getal $a+bi$, bepaal het neutraal element en het symmetrisch element in $\mathbb{C}_0, \cdot$.
\end{oefening}


\begin{oefening}
  Bereken
  \begin{enumerate}[(a)]
  \item $\displaystyle (2-i)+(3+4i)$
  \item $\displaystyle (4i-2)-(2+5i)$
  \item $\displaystyle (7+6i)+2(1+i)$
  \item $\displaystyle (2-3i)\cdot (1-i)$
  \item $\displaystyle 4i(8-2i)$
  \item $\displaystyle (2+i)(-2+i)$
  \item $\displaystyle (1-i)^2$
  \item $\displaystyle (2+3i)^2$
  \item $\displaystyle (2+i)^3$
  \end{enumerate}
\end{oefening}

\begin{oefening}
  Bereken
  \begin{enumerate}[(a)]
    \itemsep1em
  \item $\displaystyle \dfrac{2-4i}{1+i}$
  \item $\displaystyle \dfrac{7+4i}{i}$
  \item $\displaystyle \dfrac{3-2i}{3+2i}$
  \item $\displaystyle \dfrac{-1}{5+2i}$
  \item $\displaystyle \dfrac{1}{i}+1$
  \item $\displaystyle \dfrac{1-3i}{2+3i}$
  \item $\displaystyle \left(\dfrac{1+i\sqrt{2}}{1-i\sqrt{2}}\right)^2+\left(\dfrac{1-i\sqrt{2}}{1+i\sqrt{2}}\right)^2$
  \end{enumerate}
\end{oefening}

\begin{oefening}
  Geef het reëel deel en het imaginair deel van $z=\frac{i-4}{2i-3}$.
\end{oefening}

\section{Machtsverheffing}

\begin{oefening}
  Gegeven de complexe getallen:
  $$z_1=3-2i \qquad z_2=-1-i \qquad z_3=4+5i$$
  Bereken
  \begin{multicols}{2}
    \begin{enumerate}[(a)]
      \itemsep 1em
    \item $\displaystyle z_1+z_2-z_3$
    \item $\displaystyle z_1\cdot z_2$
    \item $\displaystyle \dfrac{z_1}{z_2}$
    \item $\displaystyle \overline{z_2}$
    \item $\displaystyle \dfrac{1}{\overline{z_1}}$
    \item $\displaystyle z_3\cdot \overline{z_3}$
    \item $\displaystyle z_2^2$
    \end{enumerate}
  \end{multicols}
\end{oefening}

\begin{oefening}
  Bereken
  \begin{enumerate}[(a)]
    \itemsep1em
  \item $\displaystyle \left(6+2i\right)^{-2}$
  \item $\displaystyle \left(1+2i\right)^4$
  \item $\displaystyle \dfrac{i^{35}+i^{40}}{1+i^{21}}$
  \item $\displaystyle \dfrac{(1+i)^2+(1-i)^2}{(1+i)^2-(1-i)^2}$
  \item $\displaystyle \left(\dfrac{1+i}{1-i}\right)^{30}+\left(\dfrac{1-i}{1+i}\right)^{30}$
  \end{enumerate}
\end{oefening}

\begin{oefening}
  Vereenvoudig
  \begin{enumerate}[(a)]
    \itemsep 1em
  \item $\frac{1+i}{1-i}-(1+2i)(2+2i)+\frac{3-i}{1+i}$;
  \item $2i(i-1)+\left(\overline{\sqrt(3)+i}\right)^3+(1+i)\overline{(1+i)}.$
  \end{enumerate}
\end{oefening}

\begin{oefening}
  We definiëren de {\bf positieve reële getallen} als $\mathbb{R}^+$. Dit zijn dus alle reële getallen met een positief teken, inclusief nul. We definiëren analoog de {\bf negatieve reële getallen} als $\mathbb{R}^-$. Dit zijn dus alle reële getallen met een negatief teken, inclusief nul.

  De positieve reële getallen en de negatieve reële getallen hebben dus het getal nul gemeenschappelijk:
  $$\mathbb{R}^+\cap\mathbb{R}^-=\{0\}$$

  {\bf Niet-positieve reële getallen} zijn dan alle reële getallen zonder de positieve reële getallen, {\bf niet-negatieve reële getallen} zijn dan alle reële getallen zonder de negatieve reële getallen.

  Als we een complex getal verschillend van nul vermenigvuldigen met zijn complex toegevoegde, dan krijgen we altijd een
  \begin{enumerate}[(A)]
  \item positief reëel getal.
  \item negatief reëel getal.
  \item niet-positief reëel getal.
  \item niet-negatief reëel getal.
  \end{enumerate}
\end{oefening}

\begin{oefening}*
Toon aan dat het getal
$$w=\dfrac{a}{a^2+b^2}-\dfrac{b}{a^2+b^2}i$$
de inverse voor de vermenigvuldiging is van het complex getal
$$z=a+bi\;.$$
\end{oefening}

\begin{oefening}
Bepaal $w$ zodat $(2+3i)\cdot w = 1$.
\end{oefening}

\begin{oefening}*
Bewijs volgende eigenshappen i.v.m. de complex toegevoegde van een complex getal:
\begin{enumerate}[a]
  \item $\overline{z+w}=\overline{z}+\overline{w}$
  \item $\overline{z\cdot w}=\overline{z}\cdot\overline{w}$
  \item $z\cdot\overline{z}\in\mathbb{R}$
\end{enumerate}
\end{oefening}

\pagebreak
\section{Vierkantswortel van een complex getal}

\begin{oefening}
  Bepaal alle vierkantswortels uit
  \begin{enumerate}[(a)]
    \itemsep 1em
  \item $-9$
  \item $2i$
  \item $-3-4i$
  \item $-i$
  \end{enumerate}
\end{oefening}

\begin{oefening}
  Bepaal alle vierkantswortels uit
  \begin{enumerate}[(a)]
    \itemsep 1em
  \item $(2i-3)(3-2i)$
  \item $i(1-i)$
  \item $\sqrt{2}-\sqrt{3}i$
  \item $i^5$
  \end{enumerate}
\end{oefening}

\pagebreak
\section{Vierkantsvergelijkingen in $\mathbb{C}$}

\begin{oefening}
  Los op in $\mathbb{C}$
  \begin{enumerate}[(a)]
    \itemsep 1em
  \item $z^2-6z+10=0$
  \item $z^2+(i-5)z=7i-26$
  \item $z^2-4z+13=0$
  \item $iz^2+2z-13i-16=0$
  \item $z^2-4z+7+4i=0$
  \item $z^2+(i-4)z+5(1-i)=0$
  \item $z^2-(6+i)z+7+9i=0$
  \item $z^2-(1+i)z+(1+i)=0$
  \item $z^2-(4+2i)z+3+4i=0$
  \item $z^3-2z^2+(16+8i)z=0$
  \item $z^4+(1-4i)z^2=12+16i$
  \item $z^4+z^2+1=0$
  \item $z^4+(1-i)z^2-i=0$
  \item $z^4=(1+2i\sqrt{3})z^2+2+2i\sqrt{3}$
  \end{enumerate}
\end{oefening}

\begin{oefening}
Los op in $\mathbb{C}$
  \begin{enumerate}[(a)]
    \itemsep 1em
  \item $5y+12=-\dfrac{8}{y}$
  \item $4+\dfrac{5}{x^2}=\dfrac{6}{x}$
  \item $\dfrac{3}{y-2}=\dfrac{1}{y}+1$
  \end{enumerate}
\end{oefening}

\pagebreak
\begin{oefening}*
  Bepaal $\lambda\in\mathbb{C}$ zodat
  $$iz^2+(\lambda-3i)z-2=0$$
  twee identieke wortels heeft.
\end{oefening}

\begin{oefening}*
De vierkantsvergelijking in $\mathbb{C}$
$$az^2+bz+c=0$$
met $a, b, c \in\mathbb{R}$ heeft ofwel twee (mogelijks samenvallende) reële oplossingen, ofwel twee complex oplossingen die elkaars toegevoegde zijn.
\end{oefening}

\begin{oefening}*
De vierkantsvergelijking in $\mathbb{C}$
$$z^2 + az + b = 0$$
met $a, b \in \mathbb{R}$ heeft zeker als oplossing
$$ z_1=2-i\;.$$
\begin{enumerate}[(a)]
  \item Bepaal de mogelijks andere oplossing.
  \item Bepaal de vierkantsvergelijking.
\end{enumerate}
\end{oefening}

\begin{oefening}*
Toon aan dat de vergelijking in $\mathbb{C}$
$$4z^3-6i\sqrt{3}z^2-3(3+i\sqrt{3})z-4=0$$
minstens één reële oplossing heeft.
\end{oefening}

\begin{oefening}*
Bepaal $z\in\mathbb{C}$ zodat
$$z^2+\overline{z}^2=0\;.$$
\end{oefening}

\pagebreak
\section{Meetkundige voorstelling van een complex getal}

\begin{oefening}
Stel volgende complexe getallen voor in het vlak van Gauss. Gebruik voor het complex getal $z_i$ het beeldpunt $P_i$.
$$z_1=5 \qquad z_2=1+i \qquad  z_3=2i \qquad  z_4=-2$$
$$z_5=-5+2i \qquad  z_6=3-4i \qquad  z_7=-3i \qquad  z_8=-i-2$$
\end{oefening}

\begin{oefening}
Bepaal de complexe getallen $z_i$ waarvan we in het vlak van Gauss de beeldpunten $P_i$ krijgen.
\begin{center}
\definecolor{cqcqcq}{rgb}{0.75,0.75,0.75}
\begin{tikzpicture}[line cap=round,line join=round,>=triangle 45,x=1.0cm,y=1.0cm]
\draw [color=cqcqcq,dash pattern=on 2pt off 2pt, xstep=1.0cm,ystep=1.0cm] (-3.92,-3.31) grid (4.93,3.65);
\draw[->,color=black] (-3.92,0) -- (4.93,0);
\foreach \x in {-3,-2,-1,1,2,3,4}
\draw[shift={(\x,0)},color=black] (0pt,2pt) -- (0pt,-2pt) node[below] {\footnotesize $\x$};
\draw[->,color=black] (0,-3.31) -- (0,3.65);
\foreach \y in {-3,-2,-1,1,2,3}
\draw[shift={(0,\y)},color=black] (2pt,0pt) -- (-2pt,0pt) node[left] {\footnotesize $\y$};
\draw[color=black] (0pt,-10pt) node[right] {\footnotesize $0$};
\clip(-3.92,-3.31) rectangle (4.93,3.65);
\draw (4.23,0.63) node[anchor=north west] {Re};
\draw (0.17,3.53) node[anchor=north west] {Im};
\begin{scriptsize}
\fill [color=black] (-2,2) circle (2.0pt);
\draw[color=black] (-1.74,2.25) node {$P_1$};
\fill [color=black] (0,1) circle (2.0pt);
\draw[color=black] (0.26,1.25) node {$P_2$};
\fill [color=black] (3,0) circle (2.0pt);
\draw[color=black] (3.25,0.25) node {$P_3$};
\fill [color=black] (-1,-3) circle (2.0pt);
\draw[color=black] (-0.74,-2.75) node {$P_5$};
\fill [color=black] (3,1) circle (2.0pt);
\draw[color=black] (3.25,1.25) node {$P_4$};
\end{scriptsize}
\end{tikzpicture}
\end{center}
\end{oefening}

\pagebreak
\section{Modulus en argument van een complex getal}

\begin{oefening}
Bepaal de goniometrische vorm, dus de vorm $z=r(\cos(\Phi)+i\sin(\Phi))$
\begin{multicols}{2}
\begin{enumerate}[(a)]
  \itemsep.5em
  \item $\displaystyle z=-1-i$
  \item $\displaystyle z=3i$
  \item $\displaystyle z=-2$
  \item $\displaystyle z=12-5i$
  \item $\displaystyle z=24-7i$
  \item $\displaystyle z=\cos(\alpha)-i\sin(\alpha)$
  \item $\displaystyle z=i-3$
  \item $\displaystyle z=-3i+4i-1$
  \item $\displaystyle z=\left(3+4i\right)^2$
  \item $\displaystyle z=9\cos(30\degree)+9i\sin(-30\degree)$
  \item $\displaystyle z=\dfrac{5+9i}{i}$
\end{enumerate}
\end{multicols}
\end{oefening}

\begin{oefening}
Bepaal de Carthesische vorm, dus de vorm $z=a+bi$
\begin{enumerate}[(a)]
  \itemsep.5em
  \item $\displaystyle z=2\left(\cos(45\degree)+i\sin(45\degree)\right)$
  \item $\displaystyle z=5\left(\cos(90\degree)+i\sin(90\degree)\right)$
  \item $\displaystyle z=12\left(\cos(180\degree)+i\sin(180\degree)\right)$
  \item $\displaystyle z=\pi\cdot\left(\cos(67\degree)+i\sin(247\degree)\right)$
  \item $\displaystyle z=\dfrac{13}{2}\cos(120\degree)+i\sin(120\degree)$
\end{enumerate}
\end{oefening}

\pagebreak
\section{Bewerkingen met complexe getallen in hun goniometrische gedaante}

\begin{oefening}
Bereken:
\begin{enumerate}[(a)]
  \itemsep.5em
  \item Vermenigvuldig $z_1=3\left(\cos(20\degree)+i\sin(20\degree)\right)$ met $z_2=4\left(\cos(70\degree)+i\sin(70\degree)\right)$
  \item Vermenigvuldig $z_1=2\left(\cos(50\degree)+i\sin(50\degree)\right)$ met $z_2=8\left(\cos(10\degree)+i\sin(10\degree)\right)$
  \item Vermenigvuldig $z_1=6\left(\cos(80\degree)+i\sin(80\degree)\right)$ met $z_2=2\left(\cos(20\degree)+i\sin(20\degree)\right)$
  \item Vermenigvuldig $z_1=2\left(\cos(50\degree)+i\sin(50\degree)\right)$ met $z_2=8\left(\cos(-40\degree)+i\sin(-40\degree)\right)$
  \item Vermenigvuldig $z_1=10\left(\cos(135\degree)+i\sin(135\degree)\right)$ met $z_2=5\left(\cos(45\degree)+i\sin(45\degree)\right)$
\end{enumerate}
\end{oefening}

\begin{oefening}
Bereken:
\begin{enumerate}[(a)]
  \itemsep.5em
  \item Deel $z_1=6\left(\cos(80\degree)+i\sin(80\degree)\right)$ door $z_2=2\left(\cos(20\degree)+i\sin(20\degree)\right)$
  \item Deel $z_1=2\left(\cos(50\degree)+i\sin(50\degree)\right)$ door $z_2=8\left(\cos(-40\degree)+i\sin(-40\degree)\right)$
  \item Deel $z_1=10\left(\cos(135\degree)+i\sin(135\degree)\right)$ door $z_2=5\left(\cos(45\degree)+i\sin(45\degree)\right)$
  \item Deel $z_1=3\left(\cos(20\degree)+i\sin(20\degree)\right)$ door $z_2=4\left(\cos(70\degree)+i\sin(70\degree)\right)$
  \item Deel $z_1=2\left(\cos(50\degree)+i\sin(50\degree)\right)$ door $z_2=8\left(\cos(10\degree)+i\sin(10\degree)\right)$
\end{enumerate}
\end{oefening}

\begin{oefening}
Bereken:
\begin{enumerate}[(a)]
  \itemsep.5em
  \item $\displaystyle \left(1+i\sqrt{3}\right)^4$
  \item $\displaystyle \left(\sqrt{7}-i\right)^4$
  \item $\displaystyle \left(2-2i\right)^7$
  \item $\displaystyle \left(\dfrac{\sqrt{2}}{2}-\dfrac{\sqrt{2}}{2}i\right)^4$
  \item $\displaystyle \dfrac{\left(1+i\sqrt{3}\right)^{13}}{\left(\sqrt{3}-i\right)^8}$
\end{enumerate}
\end{oefening}

\begin{oefening}
Bepaal alle
\begin{enumerate}[(a)]
  \itemsep.5em
  \item 4-de machtswortels uit $-2+i\sqrt{12}$
  \item 3-de machtswortels uit $2-2i$
\end{enumerate}
\end{oefening}

\begin{oefening}
Los op in $\mathbb{C}$
\begin{enumerate}[(a)]
  \itemsep.5em
  \item $\displaystyle z^2=1+i\sqrt{3}$
  \item $\displaystyle z^3=64$
  \item $\displaystyle z^3=4-4i\sqrt{3}$
  \item $\displaystyle z^4=-2+2i\sqrt{3}$
  \item $\displaystyle z^5=-i$
  \item $\displaystyle z^6=64$
  \item $\displaystyle z^8=1$
\end{enumerate}
\end{oefening}


\end{document}



