\documentclass[12pt,twoside]{article}

\textwidth 17cm \textheight 25cm \evensidemargin 0cm
\oddsidemargin 0cm \topmargin -2cm
\parindent 0pt
%\parskip \bigskipamount

\usepackage{graphicx}
\usepackage[dutch]{babel}
\usepackage{amssymb,amsthm,amsmath}
%\usepackage{dot2texi}
\usepackage[utf8]{inputenc}
\usepackage{nopageno}
\usepackage{pdfpages}
\usepackage{enumerate}
\usepackage{caption}
\usepackage{wrapfig}
\usepackage{pgf,tikz,pgfplots}
\pgfplotsset{compat=1.15}
\usepackage{color}
\usetikzlibrary{arrows}
\usetikzlibrary{patterns}
\usepackage{fancyhdr}
\pagestyle{fancy}
\usepackage[version=3]{mhchem}
\usepackage{multicol}
\usepackage{fix-cm}
\usepackage{setspace}
\usepackage{mhchem}
\usepackage{xhfill}
\usepackage{parskip}
\usepackage{cancel}
\usepackage{mdframed}
\usepackage{url}
\usepackage{mathtools}
\usepackage{changepage}

\newcommand{\todo}[1]{{\color{red} TODO: #1}}

\newcommand{\degree}{\ensuremath{^\circ}}
\newcommand\rad{\qopname\relax o{\mathrm{rad}}}

\newcommand\ggd{\qopname\relax o{\mathrm{ggd}}}

\pgfmathdeclarefunction{gauss}{2}{%
  \pgfmathparse{1/(#2*sqrt(2*pi))*exp(-((x-#1)^2)/(2*#2^2))}%
}

\def\LRA{\Leftrightarrow}

\newcommand{\zrmbox}{\framebox{\phantom{EXE}}\phantom{X}}
\newcommand{\zrm}[1]{\framebox{#1}}

% environment oefening:
% houdt een teller bij die de oefeningen nummert, probeert ook de oefening op één pagina te houden
\newcounter{noefening}
\setcounter{noefening}{0}
\newenvironment{oefening}
{
  \stepcounter{noefening}
  \pagebreak[0]
  \begin{minipage}{\textwidth}
  \vspace*{0.7cm}{\large\bf Oefening \arabic{noefening}}
}{%
  \end{minipage}
}

\usepackage{calc}

% vraag
\reversemarginpar
\newcounter{punten}
\setcounter{punten}{0}
\newcounter{nvraag}
\setcounter{nvraag}{1}
\newlength{\puntwidth}
\newlength{\boxwidth}
\newcommand{\vraag}[1]{
\settowidth{\puntwidth}{\Large{#1}}
\setlength{\boxwidth}{1.5cm}
\addtolength{\boxwidth}{-\puntwidth}
{\large\bf Vraag \arabic{nvraag} \addtocounter{nvraag}{1}}\vspace*{-0.5cm}
{\marginpar{\color{lightgray}\fbox{\parbox{1.5cm}{\vspace*{1cm}\hspace*{\boxwidth}{\Large{#1}}}}}
\vspace*{0.5cm}}
\addtocounter{punten}{#1}}

% arulefill
\def\arulefill{\leavevmode{\xrfill[-5pt]{0.3pt}[lightgray]\endgraf}\vspace*{0.2cm}}

% \arules{n}
\newcommand{\arules}[1]{
\color{lightgray}
%\vspace*{0.05cm}
\foreach \n in {1,...,#1}{
  \vspace*{0.75cm}
  \hrule height 0.3pt\hfill
}\color{black}\vspace*{0.2cm}}

% \arule{x}
\newcommand{\arule}[1]{
\color{lightgray}{\raisebox{-0.1cm}{\rule[-0.05cm]{#1}{0.3pt}}}\color{black}
}

% \abox{y}
\newcommand{\abox}[1]{
\fbox{
\begin{minipage}{\textwidth- 4\fboxsep}
\hspace*{\textwidth}\vspace{#1}
\end{minipage}
}
}

\newcommand{\ruitjes}[1]{
\definecolor{cqcqcq}{rgb}{0.85,0.85,0.85}
\hspace*{-2.5cm}
\begin{tikzpicture}[scale=1.04,line cap=round,line join=round,>=triangle 45,x=1.0cm,y=1.0cm]
\draw [color=cqcqcq, xstep=0.5cm, ystep=0.5cm] (0,-#1) grid (20.5,0);
\end{tikzpicture}
}


\newcommand{\assenstelsel}[5][1]{
\definecolor{cqcqcq}{rgb}{0.65,0.65,0.65}
\begin{tikzpicture}[line cap=round,line join=round,>=triangle 45,x=#1cm,y=#1cm]
\draw [color=cqcqcq,dash pattern=on 1pt off 1pt, xstep=1.0cm,ystep=1.0cm] (#2,#4) grid (#3,#5);
\draw[->,color=black] (#2,0) -- (#3,0);
%\draw[shift={(1,0)},color=black] (0pt,2pt) -- (0pt,-2pt) node[below] {\footnotesize $1$};
%\draw[color=black] (#3.25,0.07) node [anchor=south west] {$x$};
\draw[->,color=black] (0,#4) -- (0,#5);
%\draw[shift={(0,1)},color=black] (2pt,0pt) -- (-2pt,0pt) node[left] {\footnotesize $1$};
\draw[color=black] (0.09,#5.25) node [anchor=west] {\phantom{$y$}};
%\draw[color=black] (0pt,-10pt) node[right] {\footnotesize $0$};
\end{tikzpicture}
}

\newcommand{\getallenas}[3][1]{
\definecolor{cqcqcq}{rgb}{0.65,0.65,0.65}
\begin{tikzpicture}[scale=#1,line cap=round,line join=round,>=triangle 45,x=1.0cm,y=1.0cm]
\draw [color=cqcqcq,dash pattern=on 1pt off 1pt, xstep=1.0cm,ystep=1.0cm] (#2,-0.2) grid (#3,0.2);
\draw[->,color=black] (#2.25,0) -- (#3.5,0);
\draw[shift={(0,0)},color=black] (0pt,2pt) -- (0pt,-2pt) node[below] {\footnotesize $0$};
\draw[shift={(1,0)},color=black] (0pt,2pt) -- (0pt,-2pt) node[below] {\footnotesize $1$};
\draw[color=black] (#3.25,0.07) node [anchor=south west] {$\mathbb{R}$};
\end{tikzpicture}
}

\newcommand{\visgraad}[1]{\begin{tabular}{p{0.5cm}|p{#1}}&\\\hline\\\end{tabular}}

\newcommand{\tekenschema}[2]{\begin{tabular}{p{0.5cm}|p{#1}}&\\\hline\\[#2]\end{tabular}}

% schema van Horner
\newcommand{\schemahorner}{
\begin{tabular}{p{0.5cm}|p{7cm}}
&\\[1.5cm]
\hline\\
\end{tabular}}

% geef tabular iets meer ruimte
\setlength{\tabcolsep}{14pt}
\renewcommand{\arraystretch}{1.5}

\newcommand{\toets}[3]{
\thispagestyle{plain}
\vspace*{-2.5cm}
\begin{tikzpicture}[remember picture, overlay]
    \node [shift={(15.25 cm,-1.6cm)}] {%
        \includegraphics[width=1.8cm]{/home/ppareit/kaa1415/logokaavelgem.png}%
    };%
\end{tikzpicture}

\begin{tabular}{|llc|c|}
\hline
\vspace*{-0.5cm}
&&&\\
Naam & \arule{4cm} & {\Large\bf KA AVELGEM} & \\
\vspace*{-0.75cm}
&&&\\
Klas & \arule{4cm} & {\Large\bf 20...-...-...} & \\
\hline
\vspace*{-0.75cm}
&&&\\
Toets & {\bf #2} & {\large\bf #1} & Beoordeling\\
\vspace*{-0.75cm}
&&&\\
Onderwerp & \multicolumn{2}{l|}{\bf #3} &\\
\hline
\end{tabular}
}

\newcommand{\oefeningen}[1]{

\fancyhead[LE, RO]{\vspace{0.5cm} #1}
%\thispagestyle{plain}

{\bf \Large \centering Oefeningen: #1}

}

\raggedbottom

\newcommand\vl{\qopname\relax o{\mathrm{vl}}}

\newcommand\dom{\qopname\relax o{\mathrm{dom}}}
\newcommand\ber{\qopname\relax o{\mathrm{ber}}}

\newcommand\mC{\qopname\relax o{\mathrm{mC}}}
\newcommand\uC{\qopname\relax o{\mathrm{{\mu}C}}}
\newcommand\C{\qopname\relax o{\mathrm{C}}}

\newcommand\W{\qopname\relax o{\mathrm{W}}}
\newcommand\kW{\qopname\relax o{\mathrm{kW}}}
\newcommand\kWh{\qopname\relax o{\mathrm{kWh}}}


\newcommand\V{\qopname\relax o{\mathrm{V}}}
\newcommand\ohm{\qopname\relax o{\mathrm{\Omega}}}
\newcommand\kohm{\qopname\relax o{\mathrm{k\Omega}}}


\newcommand\N{\qopname\relax o{\mathrm{N}}}

\newcommand\Nperkg{\qopname\relax o{\mathrm{N/kg}}}

\newcommand\Nperm{\qopname\relax o{\mathrm{N/m}}}

\newcommand\gpermol{\qopname\relax o{\mathrm{g/mol}}}


\newcommand\kgperm{\qopname\relax o{\mathrm{kg/m}}}
\newcommand\kgperdm{\qopname\relax o{\mathrm{kg/dm}}}
\newcommand\gpercm{\qopname\relax o{\mathrm{g/cm}}}
\newcommand\gperml{\qopname\relax o{\mathrm{g/ml}}}


\newcommand{\mA}{\;\mbox{mA}}
\newcommand{\A}{\;\mbox{A}}
\newcommand{\MA}{\;\mbox{MA}}

\newcommand{\us}{\;\mu\mbox{s}}
\newcommand\s{\qopname\relax o{\mathrm{s}}}

\newcommand\h{\qopname\relax o{\mathrm{h}}}

\newcommand{\kmperh}{\;\mbox{km/h}}
\newcommand{\mpers}{\;\mbox{m/s}}
\newcommand{\kmpermin}{\;\mbox{km/min}}
\newcommand{\kmpers}{\;\mbox{km/s}}

\newcommand{\mph}{\;\mbox{mph}}

\newcommand{\Hz}{\;\mbox{Hz}}

\newcommand\Gm{\qopname\relax o{\mathrm{Gm}}}
\newcommand\Mm{\qopname\relax o{\mathrm{Mm}}}
\newcommand\km{\qopname\relax o{\mathrm{km}}}
\newcommand\hm{\qopname\relax o{\mathrm{hm}}}
\newcommand\dam{\qopname\relax o{\mathrm{dam}}}
\newcommand\m{\qopname\relax o{\mathrm{m}}}
\newcommand\dm{\qopname\relax o{\mathrm{dm}}}
\newcommand\cm{\qopname\relax o{\mathrm{cm}}}
\newcommand\mm{\qopname\relax o{\mathrm{mm}}}
\newcommand\um{\qopname\relax o{\mathrm{{\mu}m}}}
\newcommand\nm{\qopname\relax o{\mathrm{nm}}}


\newcommand\Gg{\qopname\relax o{\mathrm{Gg}}}
\newcommand\Mg{\qopname\relax o{\mathrm{Mg}}}
\newcommand\kg{\qopname\relax o{\mathrm{kg}}}
\newcommand\hg{\qopname\relax o{\mathrm{hg}}}
\renewcommand\dag{\qopname\relax o{\mathrm{dag}}}
\newcommand\g{\qopname\relax o{\mathrm{g}}}
\newcommand\dg{\qopname\relax o{\mathrm{dg}}}
\newcommand\cg{\qopname\relax o{\mathrm{cg}}}
\newcommand\mg{\qopname\relax o{\mathrm{mg}}}
\newcommand\ug{\qopname\relax o{\mathrm{{\mu}g}}}
\renewcommand\ng{\qopname\relax o{\mathrm{ng}}}

\newcommand\ton{\qopname\relax o{\mathrm{ton}}}

\newcommand\Gl{\qopname\relax o{\mathrm{Gl}}}
\newcommand\Ml{\qopname\relax o{\mathrm{Ml}}}
\newcommand\kl{\qopname\relax o{\mathrm{kl}}}
\newcommand\hl{\qopname\relax o{\mathrm{hl}}}
\newcommand\dal{\qopname\relax o{\mathrm{dal}}}
\renewcommand\l{\qopname\relax o{\mathrm{l}}}
\newcommand\dl{\qopname\relax o{\mathrm{dl}}}
\newcommand\cl{\qopname\relax o{\mathrm{cl}}}
\newcommand\ml{\qopname\relax o{\mathrm{ml}}}
\newcommand\ul{\qopname\relax o{\mathrm{{\mu}l}}}
\newcommand\nl{\qopname\relax o{\mathrm{nl}}}

\newcommand\MJ{\qopname\relax o{\mathrm{MJ}}}
\newcommand\kJ{\qopname\relax o{\mathrm{kJ}}}
\newcommand\J{\qopname\relax o{\mathrm{J}}}

\newcommand\T{\qopname\relax o{\mathrm{T}}}
\newcommand\uT{\qopname\relax o{\mathrm{{\mu}T}}}

\newcommand\grC{\qopname\relax o{\mathrm{{\degree}C}}}

\newcommand\K{\qopname\relax o{\mathrm{K}}}
\newcommand\calperK{\qopname\relax o{\mathrm{cal/K}}}

\newcommand\hPa{\qopname\relax o{\mathrm{hPa}}}
\newcommand\Pa{\qopname\relax o{\mathrm{Pa}}}

\newcommand\dB{\qopname\relax o{\mathrm{dB}}}

\newcommand\Var{\qopname\relax o{\mathrm{Var}}}

\newcommand{\EE}[1]{\cdot 10^{#1}}

\onehalfspacing

%\setlength{\headsep}{0cm}

\newenvironment{exlist}[1] %
{ \begin{multicols}{#1}
  \begin{enumerate}[(a)]
    \setlength{\itemsep}{0.5em} }
{ \end{enumerate}
  \end{multicols} }





\DeclareMathOperator*{\Bgsin}{Bgsin}
\DeclareMathOperator*{\Bgcos}{Bgcos}
\DeclareMathOperator*{\Bgtan}{Bgtan}
\DeclareMathOperator*{\Bgcot}{Bgcot}

\begin{document}

\pagestyle{fancy}
\lhead{}
\rhead{Oefeningen Rijen en Reeksen}

\section{Rijen}

\begin{oefening}
Onderzoek de convergentie van de volgende rijen:
\begin{enumerate}[(a)]
  \item $1, \frac{4}{3}, \frac{3}{2}, \ldots, \frac{2n}{n+1}, \ldots$
  \item $1, \sqrt[3]{2}, \sqrt[3]{3}, \ldots, \sqrt[3]{n}, \ldots$
  \item $\frac{1}{2}, \frac{\sqrt{3}}{2}, 1, \ldots, \sin\frac{n\pi}{6}, \ldots$
  \item $0, -\frac{1}{2}, -\frac{\sqrt{2}}{2}, \ldots, \cos\frac{n\pi}{n+1}, \ldots$
  \item $\frac{\pi}{6}, \Bgsin\frac{2}{3}, \ldots, \Bgsin\frac{n}{n+1}, \ldots$
  \item $\tan 1, 2\tan\frac{1}{2}, \ldots, n\tan\frac{1}{n}, \ldots$
  \item $4\sqrt{2}, \sqrt{11}, \ldots, \frac{2}{n}\sqrt{n^2+7}, \ldots$
  \item $\frac{1}{2}, \frac{4}{3}, \frac{9}{4}, \ldots, \frac{n^2}{n+1}, \ldots$
  \item $1, \frac{1}{\sqrt{2}}, \frac{1}{\sqrt{3}}, \ldots, \frac{1}{\sqrt{n}}, \ldots$
  \item $\frac{3}{4}, \frac{5}{8}, \frac{9}{16}, \ldots, \frac{2^n\;+\; 1}{2^{n+1}}, \ldots$
  \item $\frac{1}{2}, \frac{3}{8}, \frac{9}{26}, \ldots, \frac{3^{n-1}}{3^n\;-\;1}, \ldots$
  \item $0, -\Bgtan\frac{3}{5}, \ldots, \Bgtan\frac{1-n^2}{1+n^2}, \ldots$
  \item $4, 4\sqrt{2}, \ldots, 4n\sin\frac{\pi}{2n}, \ldots$
  \item $\frac{1}{10}, \frac{2\sqrt{2}}{13}, \ldots, \frac{n\;\sqrt{n}}{3n+7}, \ldots$
\end{enumerate}
\end{oefening}

\begin{oefening}
Van een rij worden enkele opeenvolgende termen gegeven. In welk geval kan het een rekenkundige of een meetkundige rij zijn?
\begin{enumerate}[(a)]
  \item $\ldots, 6, 13, 20, 27, \ldots$
  \item $\ldots, \frac{1}{4}, \frac{7}{12}, \frac{11}{12}, \frac{5}{4}, \ldots$
  \item $\ldots, 7, 106, 204, 303, \ldots$
  \item $\ldots, 3, -6, 12, -24, \ldots$
  \item $\ldots, 0, \frac{1}{3}, \frac{1}{6}, \frac{1}{12}, \ldots$
  \item $\ldots, 128, -32, 8, -2, \ldots$
\end{enumerate}
\end{oefening}

\begin{oefening}
Bepaal de eerste drie termen van een rekenkundige rij waarvoor:
\begin{enumerate}[(a)]
  \item $u_5=-7, u_8=-16$
  \item $u_1+u_4=16, u_3+u_5=28$
\end{enumerate}
\end{oefening}

\begin{oefening}
Bepaal de eerste drie termen van een meetkundige rij waarvoor:
\begin{enumerate}[(a)]
  \item $u_2=12, u_5=324$
  \item $u_4=24, u_6=96$
\end{enumerate}
\end{oefening}

\begin{oefening}
Bepaal vier opeenvolgende termen van een rekenkundige rij als hun som 8 is en hun product -15 is.
\end{oefening}

\section{Reeksen}

\begin{oefening}
Voor de volgende reeksen: stel een formule op voor de partiële som $s_n$, onderzoek het convergentiegedrag van de reeks, en geef de som als ze convergeren.
\begin{enumerate}[(a)]
  \item $\frac{2}{1\cdot 3} + \frac{2}{3\cdot 5} + \frac{2}{5\cdot 7} + \cdots + \frac{2}{(2n-1)(2n+1)} + \cdots$
  \item $\frac{2}{1\cdot 3} + \frac{2}{2\cdot 4} + \frac{2}{3\cdot 5} + \cdots + \frac{2}{n(n+2)} + \cdots$
  \item $\frac{1}{\sqrt{2}+1} + \frac{1}{\sqrt{3}+\sqrt{2}} + \frac{1}{2+\sqrt{3}} + \cdots + \frac{1}{\sqrt{n+1}+\sqrt{n}} + \cdots$
  \item $\frac{4\cdot2}{1^2\cdot 3^2} + \frac{4\cdot3}{2^2\cdot 4^2} + \frac{4\cdot4}{3^2\cdot 5^2} + \cdots + \frac{4(n+1)}{n^2(n+2)^2} + \cdots$
  \item $\frac{1}{1\cdot 4} + \frac{1}{4\cdot 7} + \frac{1}{7\cdot 10} + \cdots + \frac{1}{(3n-2)(3n+1)} + \cdots$
  \item $\frac{1}{1\cdot2\cdot3} + \frac{1}{2\cdot3\cdot4} + \frac{1}{3\cdot4\cdot5} + \cdots + \frac{1}{n(n+1)(n+2)} + \cdots$
  \item $\sin\frac{1}{2}\cdot\sin\frac{3}{2} + \sin\frac{1}{6}\cdot\sin\frac{5}{6} + \cdots + \sin\frac{1}{n(n+1)}\cdot\sin\frac{2n+1}{n(n+1)} + \cdots$
  \item $2\cos\frac{3}{4}\cdot\sin\frac{1}{4} + 2\cos\frac{5}{12}\cdot\sin\frac{1}{12} + \cdots + 2\cos\frac{2n+1}{2n(n+1)}\cdot\sin\frac{1}{2n(n+1)} + \cdots$
\end{enumerate}
\end{oefening}

\begin{oefening}
Onderzoek het convergentiegedrag van de volgende reeksen die alle tot een bekend type behoren. Maak gebruik van de theorie.
\begin{enumerate}[(a)]
  \item $0.1 + 0.2 + 0.3 + \cdots + 0.1\cdot n + \cdots$
  \item $1 + 1.01 + 1.01^2 + \cdots + 1.01^{n-1} + \cdots$
  \item $3 - 2 + \frac{4}{3} - \cdots + 3\cdot\left(-\frac{2}{3}\right)^{n-1}+ \cdots$
  \item $1 + 3 + 5 + \cdots + (2n-1) + \cdots$
  \item $1 + 0.99 + 0.99^2 + \cdots + 0.99^{n-1} + \cdots$
  \item $1 + \frac{1}{2\sqrt{2}} + \frac{1}{3\sqrt{3}} + \cdots + \frac{1}{n\sqrt{n}} + \cdots$
  \item $1 + \frac{1}{\sqrt{2}} + \frac{1}{\sqrt{3}} + \cdots + \frac{1}{\sqrt{n}} + \cdots$
  \item $\frac{7}{3} + 2 + \frac{5}{3} + \cdots + \frac{8-n}{3} + \cdots$
  \item $1 + \frac{\sqrt{2}}{\sqrt[10]{2^{15}}} + \frac{\sqrt{3}}{\sqrt[10]{3^{15}}} + \cdots + \frac{\sqrt{n}}{\sqrt[10]{n^{15}}} + \cdots$
\end{enumerate}
\end{oefening}

\begin{oefening}
Onderzoek het convergentiegedrag van de volgende reeksen met het kenmerk van d'Alembert:
\begin{enumerate}[(a)]
  \item $3 + \frac{3^2}{2!} + \frac{3^3}{3!} + \cdots + \frac{3^n}{n!} + \cdots$
  \item $2! + \frac{4!}{2^2} + \frac{6!}{3^2} + \cdots + \frac{(2n)!}{n^2} + \cdots$
  \item $\frac{1}{2} + \frac{2}{4} + \frac{6}{8} + \cdots + \frac{n!}{2^n} + \cdots$
  \item $\frac{1}{2} + \frac{\sqrt{2}}{4} + \frac{\sqrt{3}}{8} + \cdots + \frac{\sqrt{n}}{2^n} + \cdots$
  \item $(4 + \frac{4}{1})^{-1} + (4 + \frac{4}{2})^{-2} + (4 + \frac{4}{3})^{-3} + \cdots + (4 + \frac{4}{n})^{-n} + \cdots$
\end{enumerate}
\end{oefening}

\begin{oefening}
Onderzoek met de uitgebreide vorm van het kenmerk van d'Alembert het convergentiegedrag van de volgende reeksen:
\begin{enumerate}[(a)]
  \item $2 - \frac{3}{2!} + \frac{4}{3!} + \cdots + \frac{(-1)^{n-1}(n+1)}{n!} + \cdots$
  \item $-2 + \frac{2^2}{2} - \frac{2^3}{3} + \cdots + (-1)^n\cdot\frac{2^n}{n} + \cdots$
  \item $-\frac{1}{2} + \frac{2}{4} - \frac{3}{8} + \cdots + (-1)^n\cdot\frac{n}{2^n} + \cdots$
  \item $\frac{3}{1\cdot2} - \frac{3^2}{2\cdot3} + \frac{3^3}{3\cdot 4} + \cdots + \frac{(-1)^{n-1}\cdot 3^n}{n\cdot(n+1)} + \cdots$
  \item $\frac{1}{2} - \frac{1}{2^3\cdot 3!} + \cdots + \frac{(-1)^{n-1}}{2^{2n-1}\cdot(2n-1)!} + \cdots$
  \item $2 - \frac{2\cdot 5}{1\cdot 3} + \frac{2\cdot 5\cdot 8}{1\cdot 3\cdot 5} - \cdots + (-1)^{n-1}\cdot\frac{2\cdot 5\cdot \cdots \cdot (3n-1)}{1\cdot 3\cdot \cdots \cdot (2n-1)} + \cdots$
\end{enumerate}
\end{oefening}

\begin{oefening}
Onderzoek het convergentiegedrag met een methode naar keuze van de volgende reeksen:
\begin{enumerate}[(a)]
  \item $\frac{\sin^2\alpha}{1} + \frac{\sin^4\alpha}{2} + \frac{\sin^6\alpha}{3} + \cdots + \frac{\sin^{2n}\alpha}{n} + \cdots$
  \item $\tan\alpha + \frac{\tan^2\alpha}{\sqrt{2}} + \frac{\tan^3\alpha}{\sqrt{3}} + \cdots + \frac{\tan^n\alpha}{\sqrt{n}} + \cdots$
  \item $\frac{4!}{(1!)^2\cdot 2!} + \frac{8!}{(2!)^2\cdot 4!} + \frac{12!}{(3!)^2\cdot 6!} + \cdots + \frac{(4n)!}{(n!)^2\cdot (2n)!} + \cdots$
  \item $\frac{2!\cdot 3!}{1!\cdot 4!} + \frac{4!\cdot 6!}{2!\cdot 8!} + \frac{6!\cdot 9!}{3!\cdot 12!} + \cdots + \frac{(2n)!\cdot (3n)!}{n!\cdot (4n)!} + \cdots$
  \item $\frac{2\cdot 1!}{1} - \frac{2^2\cdot 2!}{2^2} + \frac{2^3\cdot 3!}{3^3} + \cdots + \frac{(-1)^{n-1}\cdot 2^n\cdot n!}{n^n} + \cdots$
\end{enumerate}
\end{oefening}

\end{document}


